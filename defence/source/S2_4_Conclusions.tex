%----------------------------------------------------------------------------------------
\begin{frame}
\frametitle{RB-VMS Conclusions}
\vfill
\begin{itemize}
	\onslide<1->{\item VMS formulations of NS equations can be used for the numerical simulation of turbulent flows.}
	\onslide<2->{\item Our particular VMS modelling ingredients
  	\begin{itemize}
  		\item Dynamic subscales
  		\item Nonlinear subscales
  		\item Orthgonal subscales
  	\end{itemize}
	seem to be important to simulate turbulent flows.}
	%(semidicrete stability, convergence to weak solutions).
	\onslide<3->{\item Among them dynamic and orthogonal subscales (linear or nonlinear) are the most effective.}
	\onslide<4->{\item The skewsymmetric formulation is important to keep stability.}
	%\item Further numerical examples are being carried out (Taylor Green, Channel flow).
\end{itemize}
\vfill
\end{frame}
%----------------------------------------------------------------------------------------
\begin{frame}
\frametitle{RB-VMS Limitations}
\vfill
\begin{itemize}
	\item<1-> \textbf{ASGS:}
	\begin{itemize}
		\item<2-> Poorly matrix conditioning $ \Rightarrow $ \alert{High} number of \alert{solver} iterations.
		\item<3-> No explicit projections $ \Rightarrow $ \algreen{3-}{Low} number of \algreen{3-}{nonlinear} iterations.
	\end{itemize}
	\item<4-> \textbf{OSS:}
	\begin{itemize}
		\item<5-> Better matrix conditioning $ \Rightarrow $ \algreen{5-}{Low} number of \algreen{5-}{solver} iterations.
		\item<6-> Explicit projection treatment $ \Rightarrow $ \alert{High} number of \alert{nonlinear} iterations.
	\end{itemize}
	\item<7-> \textbf{Desired:} 
		\begin{itemize}
		\item<7-> OSS with implicit projections.
		\end{itemize}
\end{itemize}
\vfill
\end{frame}
