In this section we will briefly state the basic concepts that one have to know about fluid mechanics and its mathematical description, which will be used in the forthcomming chapters.

Let us start with a review of the equations that govern the fluid flow. In this thesis we will restrict to constant-property Newtonian fluids. 

\subsection{The momentum equation}
\label{subsec-momentum}
From the Newton's second law, the momentum equation relates the fluid particle acceleration, $\frac{D\u}{Dt}$ to the external forces acting on such fluid particle. The external forces can be decomposed into the surface forces and the body forces, $\f$. The surface forces are described by the stress tensor $\boldsigma(\x,t)$ that is symmetric and, for constant-property Newtonian fluids, is defined by
\begin{equation}
\label{eq-stress_tensor_def}
\boldsigma = -\breve{p}\boldI + 2\mu\boldsymbol{\varepsilon}(\u), 
\end{equation}
where $\breve{p}$ is the pressure field, $\boldI$ the identity tensor, $\mu$ the viscosity and $\boldsymbol{\varepsilon}(\u)$ the strain rate tensor, which is defined by the following expression
\begin{equation}
\label{eq-strain_tensor_def}
\boldsymbol{\varepsilon}(\u)=\frac{1}{2}\left[\nabla\u+\left(\nabla\u\right)^T\right].
\end{equation}

The acceleration of the fluid particle is caused by the external forces according to the momentum equation
\begin{equation}
\label{eq-momentum_equation_def1}
\rho\frac{D\u}{Dt}=\nabla\cdot\boldsigma+\rho\f,
\end{equation}
being $\rho$ the density of the flow. Introducing the defintion of the stress tensor \Eq{stress_tensor_def} into \Eq{momentum_equation_def1}, and dividing by the density $\rho$, which we consider to be constant, we get an alternative expression of the momentum equation
\begin{equation}
\label{eq-momentum_equation_def2}
\frac{D\u}{Dt}=-\nabla p+\nabla\cdot(2\nu\boldsymbol{\varepsilon}(\u))+\f,
\end{equation}
where we have used that $(1/\rho)\nabla\breve{p}=\nabla(\breve{p}/\rho)=\nabla p$, being $p:=\breve{p}/\rho$ the kinematic pressure, and $\nu=\mu/\rho$ the kinematic viscosity.

\subsection{Mass conservation}
\label{subsec-mass_conservation}
Matching the rate of change of mass in a given volume and the net mass flux across the boundary of such volume, and using the divergence theorem, we get the mass conservation or continuity equation, which is given by 
\begin{equation}
\label{eq-continuity_equation_def1}
\frac{\partial\rho}{\partial t}+\nabla\cdot(\rho\u)=0,
\end{equation}
which for constant density can be simplified to the kinematic condition that the velocity field be solenoidal or, what is the same, divergence-free:
\begin{equation}
\label{eq-continuity_equation_def2}
\nabla\cdot\u=0.
\end{equation}
Equation \Eq{continuity_equation_def2} is also called incompressibility constraint, since it is a constraint on the fluid velocity, $\u$.

\subsection{Navier-Stokes equations}
\label{subsec-NS_equations}
Let $\Omega$ be a bounded domain of $\mathbb{R}^d$, where $d=2,3$ is the number of space dimensions, $\Gamma=\partial\Omega$ its boundary and $(0,T]$ the time interval. The strong form of the steady Navier-Stokes problem that govern the fluid flow motion consists of finding the velocity field $\u$ and the pressure field $p$ such that 
\begin{align}
\label{eq-NS_strong_momentum}
\frac{\partial\u}{\partial t}+\u\cdot\nabla\u-\nu\Delta\u+\nabla p=\f&\quad\mbox{in }\Omega\times(0,T],\\
\label{eq-NS_strong_continuity}
\nabla\cdot\u=0&\quad\mbox{in }\Omega\times(0,T],
\end{align}
where the decomposition of the velocity material derivative into the temporal partial derivative plus the convective derivative, $\frac{D\u}{Dt}=\frac{\partial\u}{\partial t}+\u\cdot\nabla\u$, and the fact that the velocity field is solenoidal have been used to simplify the momentum equation \Eq{momentum_equation_def2} into \Eq{NS_strong_momentum}. In forthcoming sections, the temporal partial derivative $ \frac{\partial(\cdot)}{\partial t} $ will also be denoted as $ \partial_t(\cdot) $.

Equations \Eq{NS_strong_momentum} and \Eq{NS_strong_continuity} need to be supplied with apropiate boundary and initial conditions. The boundary $\Gamma$ is divided into the Dirichlet ($\Gamma_D$) and the Neumann ($\Gamma_N$) parts such that $\Gamma_D\cup\Gamma_N=\Gamma$ and $\Gamma_D\cap\Gamma_N=\O$. Then, the boundary and intial conditions can be written as
\begin{align}
\label{eq-NS_strong_Dir}
\u&=\u_g&\mbox{on $\Gamma_D\times(0,T]$,}\\
\label{eq-NS_strong_Neu}
(-p\mathbf{I}+\nu(\nabla\u+\nabla\u^T))\cdot\mathbf{n}&=\mathbf{t}_N&\mbox{on $\Gamma_N\times(0,T]$,}\\
\label{eq-NS_strong_Ini}
\u(\x,0)&=\u_0(\x)&\mbox{in $\Omega\times\{0\}$,}
\end{align}
$\mathbf{n}$ being the unit outward vector normal to $\Gamma$. For a solid wall, the velocity field on the Dirichlet boundary $\Gamma_D$ is governed by two conditions. The first of them is the no penetration condition, the flow can not penetrate the wall.
\begin{equation}
\label{eq-NS_no_penetration}
\u\cdot\n=0\quad\mbox{on $\Gamma_D\times(0,T]$}.
\end{equation} 
For the tangential velocity components we impose the so called no-slip condition, which means that there is no relative movement between the wall and the fluid.
\begin{equation}
\label{eq-NS_no_slip}
\u\cdot\t=0\quad\mbox{on $\Gamma_D\times(0,T]$},
\end{equation}
being $\t$ a unit vector tangential to the wall. Putting together equations \Eq{NS_no_penetration} and \Eq{NS_no_slip} we have that $\u=\mathbf{0}$ on $\Gamma_D\times(0,T]$. If the wall is moving with a velocity $\u_g$ we recover the Dirichlet boundary condition \Eq{NS_strong_Dir}.

\subsection{Pressure and mass conservation}
\label{subsec-pressure_mass_conservation}
Let us now focus on the role that pressure field has in the fluid flow equations. If we take the divergence of the momentum equation \Eq{NS_strong_momentum}, assuming that the continuity equation \Eq{NS_strong_continuity} is not satisfied ($\nabla\cdot\u=\epsilon$), we have
\begin{equation}
\label{eq-NS_div_momentum}
\left(\frac{\partial}{\partial t}-\nu\Delta\right)\epsilon=-\Delta p-\nabla\cdot(\u\cdot\nabla\u).
\end{equation}
Considering problem \Eq{NS_div_momentum} with the initial condition $\epsilon_0=0$, we can say that the velocity field will be solenoidal ($\epsilon=0$) if, and only if, the following Poisson problem is satisfied
\begin{equation}
\label{eq-NS_poisson}
\Delta p=-\nabla\cdot(\u\cdot\nabla\u).
\end{equation}
Hence, we can state that the satisfaction of the Poisson problem \Eq{NS_poisson} is a necessary and sufficient condition for a solenoidal velocity field to remain solenoidal, see \cite{pope_turbulent_2000}. Furthermore, for infinite domains, the solution of \Eq{NS_poisson} using the Biot-Savart law is given by 
\begin{equation}
\label{eq-NS_poisson_solution}
p(\x,t)=\frac{1}{4\pi}\int_\Omega\frac{\nabla\cdot(\u(\mathbf{y},t)\cdot\nabla\u(\mathbf{y},t))}{|\x-\mathbf{y}|}d\mathbf{y}.
\end{equation}
An important consequence of \Eq{NS_poisson_solution} is that the pressure field is non-local, that means that a fluctuation at the point $\mathbf{y}$ affects to the hole domain. A direct repercussion of the non-locality of the pressure is that the pressure waves sent from $\mathbf{y}$ induce far-field pressure forces ($-\nabla p$) that can agitate the fluid motion at large distances from that point. Then, every part of the flow feels every other part. This consequence is more relevant in the case of turbulent fluid flows, where eddies at different locations of the flow can interact each other. 
