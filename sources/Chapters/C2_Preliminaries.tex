% PRELIMINARIES

\chapter{Preliminaries}
\label{chap-Preliminaries}

%----------------------------------------------------------------------------------------

% Define some commands to keep the formatting separated from the content 
\newcommand{\keyword}[1]{\textbf{#1}}
\newcommand{\tabhead}[1]{\textbf{#1}}
\newcommand{\code}[1]{\texttt{#1}}
\newcommand{\file}[1]{\texttt{\bfseries#1}}
\newcommand{\option}[1]{\texttt{\itshape#1}}

%----------------------------------------------------------------------------------------

\section{The Governing equations of Fluid Mechanics}
\label{sec-C2_gov_eq}
In this section we will briefly state the basic concepts that one has to know about fluid mechanics and its mathematical description, which will be used in the forthcoming chapters.

Let us start with a review of the equations that govern the fluid flow. In this thesis we will restrict to constant-property Newtonian fluids. 

\subsection{The momentum equation}
\label{subsec-momentum}
From the Newton's second law, the momentum equation relates the fluid particle acceleration, $\frac{D\u}{Dt}$, to the external forces acting on such fluid particle. The external forces can be decomposed into the surface forces and the body forces, $\f$. The surface forces are described by the stress tensor $\boldsigma(\x,t)$ that is symmetric and, for constant-property Newtonian fluids, is defined by
\begin{equation}
\label{eq-stress_tensor_def}
\boldsigma = -\breve{p}\boldI + 2\mu\boldsymbol{\varepsilon}(\u), 
\end{equation}
where $\breve{p}$ is the pressure field, $\boldI$ the identity tensor, $\mu$ the viscosity, $ \u $ the fluid velocity and $\boldsymbol{\varepsilon}(\u)$ the strain rate tensor, which is defined by the following expression
\begin{equation}
\label{eq-strain_tensor_def}
\boldsymbol{\varepsilon}(\u)=\frac{1}{2}\left[\nabla\u+\left(\nabla\u\right)^T\right].
\end{equation}

The acceleration of the fluid particle is caused by the external forces according to the momentum equation
\begin{equation}
\label{eq-momentum_equation_def1}
\rho\frac{D\u}{Dt}=\nabla\cdot\boldsigma+\rho\f,
\end{equation}
being $\rho$ the density of the flow. Introducing the definition of the stress tensor \Eq{stress_tensor_def} into \Eq{momentum_equation_def1}, and dividing by the density $\rho$, which we consider to be constant, we get an alternative expression of the momentum equation
\begin{equation}
\label{eq-momentum_equation_def2}
\frac{D\u}{Dt}=-\nabla p+\nabla\cdot(2\nu\boldsymbol{\varepsilon}(\u))+\f,
\end{equation}
where we have used that $(1/\rho)\nabla\breve{p}=\nabla(\breve{p}/\rho)=\nabla p$, being $p:=\breve{p}/\rho$ the kinematic pressure, and $\nu=\mu/\rho$ the kinematic viscosity.

\subsection{Mass conservation}
\label{subsec-mass_conservation}
Matching the rate of change of mass in a given volume and the net mass flux across the boundary of such volume, and using the divergence theorem, we get the mass conservation or continuity equation, which is given by 
\begin{equation}
\label{eq-continuity_equation_def1}
\frac{\partial\rho}{\partial t}+\nabla\cdot(\rho\u)=0,
\end{equation}
which for constant density can be simplified to the kinematic condition that the velocity field be solenoidal or, what is the same, divergence-free:
\begin{equation}
\label{eq-continuity_equation_def2}
\nabla\cdot\u=0.
\end{equation}
Equation \Eq{continuity_equation_def2} is also called incompressibility constraint, since it is a constraint on the fluid velocity.

\subsection{Navier-Stokes equations}
\label{subsec-NS_equations}
Let $\Omega$ be a bounded domain of $\mathbb{R}^d$, where $d=2,3$ is the number of space dimensions, $\Gamma=\partial\Omega$ its boundary and $(0,T]$ the time interval. The strong form of the steady Navier-Stokes problem that govern the fluid flow motion consists of finding the velocity field $\u$ and the pressure field $p$ such that 
\begin{align}
\label{eq-NS_strong_momentum}
\frac{\partial\u}{\partial t}+\u\cdot\nabla\u-\nu\Delta\u+\nabla p=\f&\quad\mbox{in }\Omega\times(0,T],\\
\label{eq-NS_strong_continuity}
\nabla\cdot\u=0&\quad\mbox{in }\Omega\times(0,T],
\end{align}
where the decomposition of the velocity material derivative into the temporal partial derivative plus the convective derivative, $\frac{D\u}{Dt}=\frac{\partial\u}{\partial t}+\u\cdot\nabla\u$, and the fact that the velocity field is solenoidal have been used to simplify the momentum equation \Eq{momentum_equation_def2} into \Eq{NS_strong_momentum}. In forthcoming sections, the temporal partial derivative $ \frac{\partial(\cdot)}{\partial t} $ will also be denoted as $ \partial_t(\cdot) $.

Equations \Eq{NS_strong_momentum} and \Eq{NS_strong_continuity} need to be supplied with appropriate boundary and initial conditions. The boundary $\Gamma$ is divided into the Dirichlet ($\Gamma_D$) and the Neumann ($\Gamma_N$) parts such that $\Gamma_D\cup\Gamma_N=\Gamma$ and $\Gamma_D\cap\Gamma_N=\emptyset$. Then, the boundary and initial conditions can be written as
\begin{align}
\label{eq-NS_strong_Dir}
\u&=\u_g&\mbox{on $\Gamma_D\times(0,T]$,}\\
\label{eq-NS_strong_Neu}
(-p\mathbf{I}+\nu(\nabla\u+\nabla\u^T))\cdot\mathbf{n}&=\mathbf{t}_N&\mbox{on $\Gamma_N\times(0,T]$,}\\
\label{eq-NS_strong_Ini}
\u(\x,0)&=\u_0(\x)&\mbox{in $\Omega\times\{0\}$,}
\end{align}
$\mathbf{n}$ being the unit outward vector normal to $\Gamma$. For a solid wall, the velocity field on the Dirichlet boundary $\Gamma_D$ is governed by two conditions. The first of them is the no penetration condition, the flow cannot penetrate the wall.
\begin{equation}
\label{eq-NS_no_penetration}
\u\cdot\n=0\quad\mbox{on $\Gamma_D\times(0,T]$}.
\end{equation} 
For the tangential velocity components we impose the so called no-slip condition, which means that there is no relative movement between the wall and the fluid.
\begin{equation}
\label{eq-NS_no_slip}
\u\cdot\t=0\quad\mbox{on $\Gamma_D\times(0,T]$},
\end{equation}
being $\t$ a unit vector tangential to the wall. Putting together equations \Eq{NS_no_penetration} and \Eq{NS_no_slip} we have that $\u=\mathbf{0}$ on $\Gamma_D\times(0,T]$. If the wall is moving with a velocity $\u_g$ we recover the Dirichlet boundary condition \Eq{NS_strong_Dir}.

\subsection{Pressure and mass conservation}
\label{subsec-pressure_mass_conservation}
Let us now focus on the role that pressure field has in the fluid flow equations. If we take the divergence of the momentum equation \Eq{NS_strong_momentum}, assuming that the continuity equation \Eq{NS_strong_continuity} is not satisfied ($\nabla\cdot\u=\epsilon$), we have
\begin{equation}
\label{eq-NS_div_momentum}
\left(\frac{\partial}{\partial t}-\nu\Delta\right)\epsilon=-\Delta p-\nabla\cdot(\u\cdot\nabla\u).
\end{equation}
Considering problem \Eq{NS_div_momentum} with the initial condition $\epsilon_0=0$, we can say that the velocity field will be solenoidal ($\epsilon=0$) if, and only if, the following Poisson problem is satisfied
\begin{equation}
\label{eq-NS_poisson}
\Delta p=-\nabla\cdot(\u\cdot\nabla\u).
\end{equation}
Hence, we can state that the satisfaction of the Poisson problem \Eq{NS_poisson} is a necessary and sufficient condition for a solenoidal velocity field to remain solenoidal, see~\cite{pope_turbulent_2000}. Furthermore, for infinite domains, the solution of \Eq{NS_poisson} using the Biot-Savart law is given by 
\begin{equation}
\label{eq-NS_poisson_solution}
p(\x,t)=\frac{1}{4\pi}\int_\Omega\frac{\nabla\cdot(\u(\mathbf{y},t)\cdot\nabla\u(\mathbf{y},t))}{|\x-\mathbf{y}|}d\mathbf{y}.
\end{equation}
An important consequence of \Eq{NS_poisson_solution} is that the pressure field is non-local, that means that a fluctuation at one point $\mathbf{y}$ affects to the hole domain. A direct repercussion of the non-locality of the pressure is that the pressure waves sent from $\mathbf{y}$ induce far-field pressure forces ($-\nabla p$) that can agitate the fluid motion at large distances from that point. Then, every part of the flow feels every other part. This consequence is more relevant in the case of turbulent fluid flows, where eddies at different locations of the flow can interact each other. 


\section{The function spaces}
\label{sec-C2_functional_spaces}
Before describing the variational formulation of Navier-Stokes equations we need an introduction to some functional spaces. In this section we define the notation used in following sections and we give some definitions of fundamental function spaces. A deeper explanation of the concepts introduced in this section can be found in any functional analysis text, for example we refer to \cite{temam}, where we find a numerical analysis for the Stokes and Navier-Stokes equations.

Let us start considering $\Omega$ to be an open set of $\mathbb{R}^n$ with boundary $\Gamma$. Otherwise stated, we assume that the boundary of $\Omega$ is locally Lipschitz. We denote by $L^p(\Omega)$, with $1<p<+\infty$ (or $L^\infty(\Omega)$), the space of real functions defined on $\Omega$ with the $p$-th power absolutely integrable (or essentially bounded real functions for the case $p=\infty$). This is a Banach space with the norm 
$$\|\u\|_{L^p(\Omega)}:=\left(\int_\Omega|\u(\x)|^pd\Omega\right)^{\frac{1}{p}}$$
(or, for $p=\infty$,
$$\|\u\|_{L^\infty(\Omega)}:=\underset{\Omega}{\mbox{ess. sup}}|\u(\x)|).$$
For $p=2$, $L^2(\Omega)$ is a Hilbert space with the scalar product
$$(\u,\v)_\Omega:=\int_\Omega\u(\x)\v(\x)d\Omega.$$
Henceforth, when considering the scalar product over all domain $\Omega$ we will exclude the subscript, reading $(\cdot,\cdot)$. Furthermore, in forthcoming sections, the $L^2(\Omega)$-norm will be simply denoted as $\|\cdot\|$.
The Sobolev space  $W^{m,p}(\Omega)$ is the space of functions in $L^p(\Omega)$ with derivatives of order less than or equal to $m$ in $L^p(\Omega)$, being $m$ an integer and with $1\leq p\leq+\infty$. This is a Banach space with the norm
$$\|\u(x)\|_{W^{m,p}(\Omega)}:\left(\sum_{j\leq m}\|D^j\u(\x)\|^p_{L^p(\Omega)}\right)^{\frac{1}{p}},$$
where $D^j$ is the differentiation operator. When $p=2$, $W^{m,2}(\Omega)=H^m(\Omega)$ is a Hilbert space with the scalar product
$$(\u,\v)_{H^m(\Omega)}:=\sum_{j\leq m}\left(D^j\u,D^j\v\right).$$
Often we are concerned about $n$-dimensional vector functions with components in one of the spaces defined above. In this case we use bold characters to denote a vectorial space
$$\Lbf^p(\Omega):=\{L^p(\Omega)\}^n,\quad\Hbf^m(\Omega):=\{H^m(\Omega)\}^n.$$

Let $\Dcal(\Omega)$ be the space of $\Ccal^\infty$ functions with compact support contained in $\Omega$. The closure of $\Dcal(\Omega)$ in $H^m(\Omega)$ is denoted by $H^m_0(\Omega)$. The space $H^m_0(\Omega)$ can be though as the space of functions that belong in $H^m(\Omega)$ that vanish on the boundary $\Gamma$ in a general sense. Functions that belong to $H^m_0(\Omega)$ satisfy the Poincaré inequality
\begin{equation}
\label{eq-Poincare_inequality}
\|u\|\leq c(\Omega)\|\nabla\u\|,\quad\forall\u\in\Hbf^m_0(\Omega).
\end{equation}

From a physical point of view, as noticed in \cite{Foias}, we can think on the space $\Lbf^2(\Omega)$ as the space of all vector fields $\u$ with finite kinetic energy. Moreover, the $\Hbf^1(\Omega)$ can be though as the space of all vector fields $\u$ with finite enstrophy.

Let us consider some additional spaces that are useful in the mathematical description of the Navier-Stokes equations. A possible way to deal with the incompressibility constrain \Eq{NS_continuity} is to consider a functional space with less regularity than $\Hbf^1(\Omega)$ defined as
$$\Hbf(div,\Omega):=\{\u\in\Lbf^2(\Omega)|\nabla\cdot\u\in\Lbf^2(\Omega)\},$$
which is a Hilbert space with the norm
$$\|\u\|_{div}:=\|\u\|+\|\nabla\cdot\u\|.$$
The closure of $\Dcal(\Omega)$ in $\Hbf(div,\Omega)$ is denoted by $\Hbf_0(div,\Omega)$.
...

\subsection{The set $\Omega$}

\subsection{$L^p$ and Sobolev spaces}

\subsection{}

\section{The variational formulation}
\label{sec-C2_variational}
\subsection{Continuous formulation}
\label{subsec-variational_continuous}
Let us consider the strong form of the Navier-Stokes problem \Eq{NS_strong_momentum}-\Eq{NS_strong_Ini}. In order to formulate the equivalent variational problem we define a set of variational spaces that incorporates the homogeneous Dirichlet boundary condition and the temporal evolution
\begin{align}
\label{eq-C2_varia_NS_u_space}
&\mathcal{V}_g:=\left\{\v\in\Hbf^1(\Omega):\left.\v\right|_{\Gamma_D}=\u_g\right\}\equiv\Hbf^1_g(\Omega),\\
\label{eq-C2_varia_NS_v_space}
&\mathcal{V}_0:=\left\{\v\in\Hbf^1(\Omega):\left.\v\right|_\Gamma=0\right\}\equiv\Hbf^1_0(\Omega),\\
\label{eq-C2_varia_NS_p_space}
&\mathcal{Q}:=\Lbf^2(\Omega)/\mathbb{R}.
\end{align}
Given sufficiently smooth functions $ \v\in\mathcal{V}_0 $ and $ q\in\mathcal{Q} $, we obtain the variational or weak version of the Navier-Stokes equations multiplying \Eq{NS_strong_momentum} by $ \v $  and \Eq{NS_strong_continuity} by $ q $, integrating over $ \Omega $ and integrating by parts the second order derivatives. Then the variational Navier-Stokes problem reads: find $ \u\in\Lbf^2(0,T;\mathcal{V}_g),\\ $ and $ p\in\Lbf^1(0,T;\mathcal{Q}) $ such that: 
\begin{align}
\label{eq-C2_varia_NS_weak_momentum}
(\partial_t\u,\v)+\left(\nu\left(\nabla\u+\nabla\u^T\right),\nabla\v\right)+b(\u,\u,\v)+(\nabla p,\v)=&\left\langle f,\v\right\rangle&\forall\v\in\mathcal{V}_0,\\
\label{eq-C2_varia_NS_weak_continuity}
(\nabla\cdot\u,q)=&\ 0&\forall q\in\mathcal{Q}.
\end{align}
Adding up equations \Eq{C2_varia_NS_weak_momentum}-\Eq{C2_varia_NS_weak_continuity} we obtain an alternative weak form of the incompressible Navier-Stokes problem \Eq{NS_strong_momentum}-\Eq{NS_strong_Ini} consists, e.g., in finding $[\u,p]\in {L}^2(0,T;\mathcal{V}_g)\times {\cal D}'(0,T;\mathcal{Q})$ (distributions in time with values in $\mathcal{Q}_0$) such that
\begin{equation}
\label{eq-C2_NS_weak}
(\partial_t\u,\v) + B(\u;[\u,p],[\v,q]) = \left<\f,\v\right> 
\quad\quad\forall\v\in\mathcal{V}_0,\quad\forall q\in\mathcal{Q},
\end{equation}
satisfying the initial condition \Eq{NS_strong_Ini} in a weak sense. Here the form $B({\a};[\u,p],(\v,q))$ is defined as 
\begin{equation}
\label{eq-C2_bilinear}
B(\a;[\u,p],[\v,q]):=\nu(\nabla\u,\nabla\v)+b(\a,\u,\v)-(p,\nabla\cdot\v)+(q,\nabla\cdot\u)
\end{equation}
where the trilinear weak form of the convective term $b(\u,\v,\w)$ can be written in the following three equivalent ways
\begin{align}
\label{eq-C2_b_noskew}
&b(\u,\v,\w)=(\u\cdot\nabla\v,\w)&&\mbox{Non conservative},\\
\label{eq-C2_b_skew1}
&b(\u,\v,\mathbf{w})=\frac{1}{2}(\u\cdot\nabla\v,\mathbf{w})-\frac{1}{2}(\v,\u\cdot\nabla\mathbf{w})&&\mbox{Skew-symmetric (type 1)},\\
\label{eq-C2_b_skew2}
&b(\u,\v,\mathbf{w})=(\u\cdot\nabla\v,\mathbf{w})+\frac{1}{2}(\v\cdot\mathbf{w},\nabla\cdot\u)&&\mbox{Skew-symmetric (type 2)}.
\end{align}
Note that in the trilinear weak forms \Eq{C2_b_noskew}-\Eq{C2_b_skew2} the boundary integral terms that arise from the integration by parts have been neglected. This assumption is valid when strong Dirichlet boundary conditions are considered over all the boundary. Despite of that, this equivalence is lost at the discrete level. The skew-symmetric form (type 2) (\ref{eq-C4_b_skew2}) is very common when numerical analysis are presented ~\cite{badia_convergence_2014,burman_galerkin_2009,guermond_faedogalerkin_2007} but the skew-symmetric form (type 2) (\ref{eq-C4_b_skew1}) has important advantages when the first argument is a discontinuous function, as will be shown in forthcoming chapters.

The well-posedness of problem \Eq{C2_NS_weak} relies on the called LBB condition, which stands for the name of the authors that developed works related to that condition. See the works by Ladyzhenskaya ~\cite{ladyzhenskaya1969mathematical}, Babu\^{s}ka ~\cite{babuska_error-bounds_1971} and Brezzi ~\cite{brezzi1974existence}. The LBB condition is also called \textit{inf-sup} condition and reads as follows: there exist a positive constant $ \beta $ such that,
\begin{equation}
\label{eq-C2_lbb}
\inf_{q\in\mathcal{Q}}\sup_{\v\in\mathcal{V}_0}\frac{(\nabla\cdot\v,q)}{\|\v\|_\mathcal{V}\|q\|_{\mathcal{Q}/\ker A^t}}\ge\beta>0,
\end{equation}
being $ A^t $ the adjoin of the operator defined as 
\begin{equation}
\label{eq-C2_lbb_operator}
A: \mathcal{V}_0\rightarrow\mathcal{Q'} \quad| \quad\langle A(\v),a\rangle_{\mathcal{Q'}\times\mathcal{Q}}=(\nabla\cdot\v,q)\quad\quad\forall\v\in\mathcal{V}_0,\quad\forall q\in\mathcal{Q}
\end{equation}

\subsection{The Finite Element method}
\label{subsec-variational_finite_element}
In order to approximate the solution of the variational problem \Eq{C2_NS_weak}, one needs to construct finite-dimensional spaces in which the solution can be computed. The approach followed in this work to construct such finite-dimensional spaces is the called Finite Element (FE) method. According to Ciarlet, see ~\cite{ciarlet_finite_1978}, we can define a FE as follows.

Let $ K\subseteq\Rbb^n $ be a bounded closed set with nonempty interior and piece-wise smooth boundary, the element domain. Let $ \mathcal{S} $ be a finite-dimensional space of functions on $ K $, the space of shape functions. Let $ \mathcal{N}=\{\mathcal{N}_1,\mathcal{N}_2,...,\mathcal{N}_k\} $ be a basis for $ \mathcal{S}' $, the set of nodal variables. Then, $ (K,\mathcal{S},\mathcal{N}) $ is called a FE.

We refer to Brenner et al ~\cite{brenner_mathematical_2007} for a deeper explanation of the FE definitions.

In this thesis we will mainly use FE spaces composed by quadrilateral finite elements built from a tensor product of polynomials. For the 3D case, we consider a reference FE $ (\widetilde{K},\widetilde{\mathcal{S}},\widetilde{\mathcal{N}}) $ with $ \widetilde{K} $ a cube defined in $ [-1,1]^3 $, $ \widetilde{\mathcal{S}}=Q_k $ being
$$ Q_k:=\left\{\sum_jc_jp_j(x)q_j(y)r_j(z):\mbox{ with $ p_j $, $ q_j $ and $ r_j $ polynomials of degree $j \leq k $}\right\}, $$
and $ \widetilde{\mathcal{N}} $ denoting the point evaluations at $ \left\{(t_l,t_m,t_n):\ l,m,n=0,1,...,k \right\} $ where \\$\left\{-1 = t_0 < t_1 < ... < t_k = 1 \right\}$.

Let us now consider a FE partition $ \mathcal{T}_h $ of the domain $ \Omega $ composed by a set of elements $ \left\{K_e\right\}_{e=1}^{ne} $, being $ ne $ the total amount of elements in the domain. Let us consider $ F_K $ a mapping from $ K $ to $ \widetilde{K} $, i. e. $ F_K(K)=\widetilde{K} $ with its pull-back map defined as $ F^*_K(\hat{f}):=\hat{f}\circ F_K $, see ~\cite{ciarlet_general_1972,brenner_mathematical_2007} for more details on equivalence between FE.

Then the FE spaces for the velocity and pressure fields equivalent to \Eq{C2_varia_NS_u_space}-\Eq{C2_varia_NS_p_space} can be defined as
\begin{align}
\label{eq-C2_varia_NS_vh_space}
&\mathcal{V}_{h}:=\left\{\v_h\in(\mathcal{C}^0(\Omega))^d:\left.\v_h\right|_K=\tilde{\v}\circ F_K^{-1},\ \tilde{\v}\in(Q_{k_v})^d,\ K\in\mathcal{T}_h\right\},\\
\label{eq-C2_varia_NS_vgh_space}
&\mathcal{V}_{g,h}:=\left\{\v_h\in\mathcal{V}_h:\left.\v_h\right|_{K\cap\Gamma}=\u_g\right\},\\
\label{eq-C2_varia_NS_v0h_space}
&\mathcal{V}_{0,h}:=\left\{\v_h\in\mathcal{V}_h:\left.\v_h\right|_{\partial K}=0\right\},\\
\label{eq-C2_varia_NS_qh_space}
&\mathcal{Q}_h:=\left\{\mathcal{C}^0(\Omega)\cap\Lbf^2(\Omega)/\mathbb{R}:\left.q_h\right|_K=\tilde{q}\circ F_K^{-1},\ \tilde{q}\in Q_{kq},\ K\in\mathcal{T}_h \right\}.
\end{align}
Where $ k_v $ and $ k_q $, not necessarily equal, are the degree of the polynomials used to define the interpolation space for the velocity and pressure fields, respectively. In what follows, the subindex $ h $ will denote functions related to the FE space. Note that in this work both velocity and pressure field spaces, $ \mathcal{V}_h $ and $ \mathcal{Q}_h $, are considered to be made by continuous functions in the same partition of the domain, $ \mathcal{T}_h $.

\subsection{Semi-discrete formulation}
\label{subsec-variational_semidiscrete}
Let us consider a FE partition $ \mathcal{T}_h $ of the domain $ \Omega $ from which we can construct conforming finite dimensional spaces for the velocity $\mathcal{V}_{g,h}\subset\mathcal{V}_g$, and for the pressure $ \mathcal{Q}_{0,h}\subset\mathcal{Q}_0 $. The spaces $\mathcal{V}_{g,h}$ and $ \mathcal{Q}_{0,h} $ are the ones defined in the previous section, equations \Eq{C2_varia_NS_vgh_space} and \Eq{C2_varia_NS_qh_space} respectively.

The Galerkin FE approximation of \Eq{C2_NS_weak} consists in finding $[\u_h,p_h]\in {L}^2(0,T;\mathcal{V}_{g,h})\times {\cal D}'(0,T;\mathcal{Q}_h)$ such that
\begin{equation}
\label{eq-C2_NS_weak_discrete}
(\partial_t\u_h,\v_h) + B(\u_h;[\u_h,p_h],[\v_h,q_h]) = \left<\f,\v_h\right> 
\quad\quad\forall\v_h\in\mathcal{V}_{0,h},\quad\forall q_h\in\mathcal{Q}_h,
\end{equation}

Problem \Eq{C2_NS_weak_discrete} is well posed if the discrete \textit{inf-sup} condition equivalent to \Eq{C2_lbb} is satisfied. The discrete version reads: there exist a positive constant $ \beta_d $, independent of $ h $, such that,
\begin{equation}
\label{eq-C2_lbb_discrete}
\inf_{q_h\in\mathcal{Q}_h}\sup_{\v_h\in\mathcal{V}_{0,h}} \frac{(\nabla\cdot\v_h,q_h)}{\|\v_h\|_{\mathcal{V}_h}\|q\|_{\mathcal{Q}_h/\ker A_h^t}}\ge\beta_d>0,
\end{equation}
with $ A_h $ the equivalent operator to the one defined in \Eq{C2_lbb_operator}.



\section{The Variational Multiscale method}
\label{sec-C2_vms}
Let us consider a FE partition $\mathcal{T}_h$ of the domain $\Omega$ from which we can construct conforming finite dimensional spaces for the velocity $\mathcal{V}_{0,h} \subset \mathcal{V}_0$, and for the pressure $\mathcal{Q}_{0,h}\subset \mathcal{Q}_0$. 

It is well known that the Galerkin FE approximation \Eq{NS_galerkin} has numerical instabilities for high mesh Reynolds number problems, i.e., when the nonlinear convective term dominates the viscous term. Another drawback of that formulation is the discrete \textit{inf-sup} condition that must be satisfied by the pair $\mathcal{V}_{0,h} \times\mathcal{Q}_{0,h}$ in order to have a well-posed problem with bounded pressure. These difficulties are overcome by using the VMS approach, introduced by Hughes in \cite{hughes,hughes}, and that is stated as follows.

Let us consider a two-scale decomposition of spaces $\mathcal{V}_0$ and $\mathcal{Q}_0$ such that $$\mathcal{V}_0=\mathcal{V}_{0,h}\oplus\widetilde{\mathcal{V}}_0$$ and $$\mathcal{Q}_0=\mathcal{Q}_{0,h}\oplus\widetilde{\mathcal{Q}}_0,$$ where $\widetilde{\mathcal{V}}_0$ and $\widetilde{\mathcal{Q}}_0$ are infinite-dimensional spaces that complete the FE spaces in $\mathcal{V}_0$ and $\mathcal{Q}_0$, respectively. Hereinafter the subscript $(\cdot)_h$ will denote the FE component and the tilde $\widetilde{(\cdot)}$ the subgrid component. Applying the two-scale decomposition to \Eq{NS_weak} we obtain a discrete problem
\begin{align}
\label{eq-C2_vms_NS_discrete}
(\partial_t\u_h,\v_h)+(\partial_t\tilde{\u},\v_h)&+B(\a;[\u_h,p_h],[\v_h,q_h])\\\nonumber
&+\left(\tilde{\u},\mathcal{L}_{\a}^*(\v_h,q_h)\right)_h-\left(\tilde{p},\nabla\cdot\v_h\right)=\left<\f,\v_h\right>,
\end{align}
where $(\cdot,\cdot)_h=\sum_{K\in\mathcal{T}_h}(\cdot,\cdot)_K$ is the sum of scalar products (\ref{eq:scalar_product}) over each element $K$ of the partition $\mathcal{T}_h$, and
\begin{equation}
\label{eq-C2_vms_adjoint}
\mathcal{L}_{\a}^*(\v_h,q_h):=-\nu\nabla^2\v_h-\a\cdot\nabla\v_h-\nabla q_h
\end{equation}
is the formal of the adjoint operator of the momentum equation. The term involving the adjoint operator comes from an elementwise integration by parts of the terms involving the subscales, in which the boundary terms 
$\left( \v_{h},\nu \n\cdot \nabla \tilde{\u}\right)_{\partial h}$ and
$\left( q_{h},\n\cdot \tilde{\u}\right)_{\partial h}$
have been neglected (the subscript ${\partial h}$ is used to denote the sum over all elements of the integral on the boundary of each element). It also involves the approximation 
$b(\a,\tilde{\u},\u_h) \approx -(\tilde{\u},\a\cdot\nabla\v_h)$
which implies neglecting 
$\left( \v_{h},\n\cdot \a \tilde{\u}\right)_{\partial h}$ and
$(\tilde{\u},\nabla\cdot\a\v_h)$. 
These approximations are discussed in  \cite{codina_time_2007} together with the choice of $\a$ which defines the type of scale splitting (linear or nonlinear), also discussed below.

The discrete problem depends on $\tilde{\u} \in \widetilde{\mathcal{V}}_0$ and on $\tilde{p}\in \widetilde{\mathcal{Q}}_0$,  $\widetilde{\mathcal{V}}_0$ and $\widetilde{\mathcal{Q}}_0$ being infinite-dimensional. Therefore, the equations for $\tilde{\u}$ and $\tilde{p}$ obtained after applying the two-scale decomposition cannot be directly solved, but some modeling steps are needed to obtain a feasible method. Considering the subscale as a time-dependent variable of the problem (see below) and approximating the Navier-Stokes operator by two stabilization parameters $\tau_m^{-1}$ and $\tau_c^{-1}$ (see for example \cite{codina_time_2007}), the fine scale problem can be written as
\begin{align}
\label{eq-C2_vms_velo_sgs}
\partial_t\tilde{\u}+\tau_m^{-1}\tilde{\u}=\mathcal{P}(\R_u),\\
\label{eq-C2_vms_press_sgs}
\tau_c^{-1}\tilde{p}=\mathcal{P}(R_p).
\end{align}
In \Eq{velo_sgs}-\Eq{press_sgs} $\mathcal{P}$ denotes the projection onto the space of subscales, which is discussed below. In turn, the vector $\R$ is the residual of the Navier-Stokes equations \Eq{NS_strong_mome}-\Eq{NS_strong_inc}, defined as $\R=[\R_u,R_p]^T$, with
\begin{align}
\label{eq-C2_vms_Ru}
%&\R_u=\f-\left(\partial_t\u_h+N((\u_h+\tilde{\u}),\u_h)+\nabla p_h \right),\\
%&\R_u=\f-\left(\partial_t\u_h+ \a \cdot \nabla \u_h + \nabla p_h \right),\\
&\R_u=\f-\partial_t\u_h-\mathcal{L}_{\a}(\u_h,p_h),\\
\label{eq-C2_vms_Rp}
&R_p=-\nabla\cdot\u_h.
\end{align}
where
\begin{equation}
\label{eq-C2_vms_operator}
\mathcal{L}_{\a}(\v_h,q_h):=-\nu\nabla^2\v_h+\a\cdot\nabla\v_h+\nabla q_h
\end{equation}
%and $N((\u_h+\tilde{\u}),\u_h)=(\u_h+\tilde{\u})\cdot\nabla\u_h$ is the convective term. 
Finally, the expressions of the stabilization parameter $\tau_m$ is 
\begin{align}
\label{eq-C2_vms_tau_m}
&\tau_m=\left(\frac{c_1\nu}{h^2}+\frac{c_2|\a|}{h}\right)^{-1},
\end{align}
whereas we consider two possible definitions of $\tau_c$, viz. $\tau_c = 0$ (which implies $\tilde{p} = 0$) and 
\begin{align}
\label{eq-C2_vms_tau_c}
&\tau_c=\frac{h^2}{c_1\tau_m},
\end{align}
where $h$ is the mesh size and $c_1$ and $c_2$ are algorithmic constants. Let us comment on expression \Eq{tau_m}:
\begin{itemize}
\item The influence of the constants $c_1$ and $c_2$ is discussed in Section XXXX. A theoretical way to determine them would be to impose that the numerical dissipation they introduce be equal to the molecular dissipation in turbulent regimes, as explained in~\cite{guasch-codina-13}. 
\item The definition of $\tau_m$ in \Eq{tau_m} is not standard, in the sense that the one used often depends on the time step size of the time discretization, $\delta t$. Instead of \Eq{tau_m}, $\tau_m^{-1}=\frac{1}{\delta t} + \frac{c_1\nu}{h^2}+\frac{c_2|\a|}{h}$ is more often considered (see, e.g., \cite{Hsu2010,gamnitzer_time-dependent_2010}). We refer to 
Section XXXX~\ref{sec:small_time_step} for a more detailed discussion about this topic. Likewise, other expressions with the same asymptotic behavior in terms of $h$, $\nu$ and $|\a|$ can also be employed.
\item Expression \Eq{tau_m} corresponds to linear isotropic elements. If elements of order $p$ are used ($p$ is not the pressure, here), $c_1$ must be replaced by $c_1 p^4$ and $c_2$ by $c_2 p$. For anisotropic elements, the definition of $h$ within each element is not obvious. A possibility is explained in \cite{Principe2010}.
\end{itemize}

In the following three sections we discuss the particular ingredients of our VMS models. A different summary can also be found in \cite{Codina-chap-2011}, together with some numerical experiments. 

\subsection{The dynamics of the subscales}
\label{subsec-C2_vms_dyn}
Stabilized formulations were originally developed for steady convection-diffusion \cite{Brooks_1982} and Stokes \cite{Douglas_1989,Hughes_1986_5} problems. As the numerical instabilities have a spatial nature, the time dependency of the subscales was not considered, and the standard choice \cite{Hughes2000,hughes_large_2001,bazilevs_variational_2007} was to take 
\begin{equation}
\label{eq-C2_vms_static_sgs}
\tilde{\u}=\tau_m \mathcal{P}(\R_u),
\end{equation}
that is, to neglect the temporal derivative of the subscales in (\ref{eq:velo_sgs}). In this case, the subscales are called quasi-static in what follows.

The subscale as a time dependent variable of the problem was introduced in \cite{codina_stabilized_2002,codina_time_2007}. It gives rise to important properties like commutativity of space and time discretization, stability without restrictions on the time step size \cite{codina_time_2007,Badia2009a} and, combined with orthogonal subscales, to convergence towards weak solutions of the Navier-Stokes equations \cite{Badia2013Convergence} and the possibility of predicting backscatter \cite{Codina-chap-2011,Principe2009}. 

Equation \Eq{velo_sgs} can be analytically integrated to give
\begin{equation}
\label{eq:exact_sgs}
\tilde{\u}(t^*)=\tilde{\u}(0) + \mu^{-1}(t^*) \int_0^{t^*} \mu(t) \mathcal{P}\R_u {\rm d}t, \quad \quad \mu(s)=\exp \int_0^s \tau^{-1}(t){\rm d}t,
\end{equation}
where it is explicitly seen that the subscale is a function of the residual but also of the flow history. In practice this integration is performed numerically, as described below.

\subsection{(Non)linear scale splitting}
\label{subsec-C2_vms_nl}
The original VMS formulation \cite{hughes_multiscale_1995,hughes_variational_1998} was developed having linear problems in mind and its extension to the Navier-Stokes equations was implicitly based on a ``linearization'', fixing the advection velocity and applying the multiscale splitting to the rest of the terms. A nonlinear scale splitting was used in \cite{Hughes2000,hughes_large_2001} together with an explicit resolution of the small scales in which a Smagorinsky damping was introduced. A nonlinear scale splitting with modeled subscales was used in \cite{codina_stabilized_2002, bazilevs_variational_2007} and in \cite{codina_time_2007}, where it was shown that it leads to global conservation of momentum. We therefore consider both options
%The vector $\a$ appearing in (\ref{eq:bilinear}) and (\ref{eq:adjoint}) is the advection velocity of the convective term. It can be defined in two different ways depending if we use a linear or a nonlinear scheme for the subscales definition. That is, 
\begin{align}
\label{eq-C2_vms_a_lin}
&\a=\u_h&&\mbox{for linear subscales},\\
\label{eq-C2_vms_a_nl}
&\a=\u_h+\tilde{\u}&&\mbox{for nonlinear subscales}.
\end{align}

\begin{remark}
\label{rem-skewsym}
When we use the nonlinear definition for the advection velocity, $\a=\u_h+\tilde{\u}$, the skew-symmetric term \textit{type 2} \Eq{b_skew2} in the FE equation \Eq{NS_discrete} reads: 
\begin{equation}
\label{eq-C2_vms_b_skew2_aux1}
b(\a,\u_h,\v_h)=((\u_h+\tilde{\u})\cdot\nabla\u_h,\v_h)+\frac{1}{2}(\u_h\cdot\v_h,\nabla\cdot\u_h)+\frac{1}{2}(\u_h\cdot\v_h,\nabla\cdot\tilde{\u}).
\end{equation}
The last term is not well-defined, since it includes derivatives of the discontinuous subscale $\tilde{\u}$. One possibility is to neglect it (as previously done with other similar terms when arriving to \Eq{NS_discrete}), which implies
\begin{equation}
b(\a,\u_h,\u_h)=-\frac{1}{2}(|\u_h|^2,\nabla\cdot\tilde{\u}),
\end{equation}
the same result obtained when the non conservative form is used.
By contrast, the skew-symmetric term \textit{type 1} in the FE equation \Eq{NS_discrete} reads
\begin{equation}
\label{eq-C2_vms_b_skew1_aux1}
b(\a,\u_h,\v_h)=\frac{1}{2}((\u_h+\tilde{\u})\cdot\nabla\u_h,\v_h)-\frac{1}{2}((\u_h+\tilde{\u})\cdot\u_h,\nabla\cdot\v_h)
\end{equation}
from where
\begin{equation}
b(\a,\u_h,\u_h)=0.
\end{equation}
% and we can not use the FE shape function derivatives to approximate it because the subscales are on another space. Then we avoid this problem by neglecting this term. 
%Therefore, the skew-symmetric term \textit{type 2} implementation for a nonlinear definition of the subscales will be
%\begin{equation}
%\label{eq:b_skew2_aux2}
%b(\a,\u_h,\v_h)=(\a\cdot\nabla\u_h,\v)+\frac{1}{2}(\u_h\cdot\v_h,\nabla\cdot\a)-\frac{1}{2}(\u_h\cdot\v_h,\nabla\cdot\tilde{\u}).
%\end{equation}
In Subsection XXXX(\ref{subsec:result_DHIT}) we will see the influence of the two forms of the convective term on the results.
It is worth noting that the same approximations have been introduced in all cases to implement $b(\a,\tilde{\u},\u_h)$, but these approximations are taken into account in the (usual) energy estimates of Section XXXX\ref{sec:energy}.
%Note that \textit{type 2} is the standard choice.
\end{remark}

\begin{remark}
At the continuous level, the different expressions of the convective term are also equivalent to the so called conservation form
\begin{align}
b(\u,\v,\w)= -(\u\otimes \v,\nabla\w).\nonumber
\end{align}
In the discrete problem, the nonlinear scale splitting leads to the following terms in the momentum equation:
\begin{align}
b(\a,\u_h +\tilde\u,\v_h)= - (\u_h\otimes\u_h, \nabla\v_h)
- (\u_h\otimes\tilde\u, \nabla\v_h)- (\tilde\u\otimes\u_h, \nabla\v_h)
- (\tilde\u\otimes\tilde\u, \nabla\v_h).\label{eq:nonconconv}
\end{align}
Even if this is not exactly what we get using the non-conservative or skew-symmetric forms because of the approximation error, this allows us to interpret the different contributions arising from the nonlinear scale splitting. As it is explained in \cite{Codina-chap-2011}, from (\ref{eq:nonconconv}) we can identify the contributions from the cross stresses, the Reynolds stresses and the subgrid scale tensor. 
\end{remark}

\subsection{The space for the subscales}
\label{subsec-C2_vms_oss}
The selection of the space for the approximation of the subscales determines the projection $\mathcal{P}$ appearing in the right-hand side of \Eq{velo_sgs} and \Eq{press_sgs}. The first option, already considered in \cite{Hughes2000,hughes_large_2001,bazilevs_variational_2007} and named Algebraic Subgrid Scale (ASGS) in \cite{codina_stabilization_2000} is to take the subscales in the space of the residuals, that is,
%method is characterized by the following projection definition:
\begin{equation}
\label{eq-C2_vms_P_ASGS}
\mathcal{P}:=\mathbf{I}.
\end{equation}
Another possibility introduced in \cite{codina_stabilization_2000} is to consider the space of the subscales orthogonal to the FE space. The main motivation of the method is that a stability estimate for the projection onto the FE space of the pressure and/or the convective terms can already be obtained in the standard Galerkin method and therefore the only ``missing'' part is the orthogonal one. The Orthogonal Subscales (OSS) method is then characterized by the following projection definition:
\begin{equation}
\label{eq-C2_vms_P_OSS}
\mathcal{P}:=\Pi_h^\bot=\mathbf{I}-\Pi_h,
\end{equation}
where $\Pi_h$ is the projection onto the FE space. With this choice, the residual of the momentum equation does not depend on $\partial_t\u_h$. Likewise, $\mathcal{P}(\f)$ in this case is only well defined for $\f\in L^2(\Omega)^d$. In the case of minimum regularity, $\f\in H^{-1}(\Omega)^d$, this term can be simply neglected without upsetting the accuracy of the method.

In fact, with this choice, the orthogonality between the space of subscales and the FE space is only guaranteed when the stabilization parameters are constant. If this is not the case, the method is still optimally convergent \cite{Codina_2008a} but this property is lost. In order to have truly orthogonal subscales, \emph{which guarantees a proper separation of the FE and the subgrid scale kinetic energies} (see below and Section \ref{sec:energy}) a slight modification of the projection $\Pi_h$ is needed (see \cite{Codina_2008a}). We will use two different weighted projections: one for the velocity subscales ($\Pi_m$) in \Eq{velo_sgs} and another for the pressure subscales ($\Pi_c$) in \Eq{press_sgs}. We define the weighted projections $\Pi_m$ and $\Pi_c$ such that given any vector $\w\in\mathcal{V}_0$ and any scalar $r\in\mathcal{Q}_0$ we have
\begin{align}
\label{eq-C2_vms_Pi_tau_m}
&(\tau_m\Pi_m(\w),\v_h)=(\tau_m\w,\v_h)&&\forall\v_h\in\mathcal{V}_{0,h},\\
\label{eq-C2_vms_Pi_tau_c}
&(\tau_c\Pi_c(r),q_h)=(\tau_cr,q_h)&&\forall q_h\in\mathcal{Q}_{0,h}.
\end{align}
These definitions guarantee the orthogonality between the FE and subscale spaces in the case of static subscales, that is, neglecting temporal derivatives in \Eq{velo_sgs}. It then follows that the term containing the temporal derivative of the subscale in the FE equation \Eq{NS_discrete} also vanishes.

However, if the dynamic version of the method is used, the weight of the projection \Eq{Pi_tau_m} must be conveniently modified to ensure the mentioned orthogonality. As it can be seen in XXXXX(\ref{eq:exact_sgs}), the definition of the weight depends on the time integration strategy, as explicitly stated in Section XXXX\ref{sec:discrete}.

\section{Time integration}
\label{sec-C2_time_integration}
In this section, our aim is to state some basic concepts about the time discretization. Once we have defined the semi-discrete problem, as it has been formulated in Section \ref{subsec-variational_semidiscrete}, we end up with an Ordinary Differential Equation (ODE), which has to be integrated in order to get the solution of the problem.

In the current work we only will consider the called \textit{direct integration} methods. The \textit{direct integration} of the transient equations rely on a numerical step-by-step procedure, where the word \textit{direct} means that no transformation of the ODE problem is carried out a priori. Looking at the litarature, many techniques can be found based in this kind of procedure. For instance in \cite{bathe_finite_2006} the description of the following methods can be found: the central difference method, the Houbold method, the Newmark method or the $ \theta $-method. An exhaustive analysis of such methods can be found in \cite{belytschko_computational_1983}.

Other commonly used time integration methods in the computational fluid dynamics field are the Backward Differentiation Formulas (BDF), the generalized-$ \alpha $ method or the Runge-Kutta time integration schemes, see \cite{brayton_new_1972, jansen_generalized-alpha_2000, dettmer_analysis_2003,hairer_solving_2008} for instance.

Within the forthcoming chapters, only the $ \theta $-methods and Runge-kutta schemes are used to integrate in time the incompressible Navier-Stokes equations. Consequently, in the following subsections only these two methods are described. 

\subsection{$ \theta $-method}
Let suppose that we have an initial-value problem with first order ODE of the form
\begin{align}
\label{eq-C2_time_ODE}
&\frac{\partial u}{\partial t}=f(t,u),\quad\in(0,T)\\
\label{eq-C2_time_ODE_0}
&u(0)=u_0.
\end{align}
One of the most popular, widely used and simplest method to solve problem \Eq{C2_time_ODE}-\Eq{C2_time_ODE_0} are the so-called single step (one-step) schemes, particularly, the theta-method, which is usually denoted as $ \theta $-method. Using the notation $u(t_n) = u_n$, the $ \theta $-method is defined as
\begin{align*}
\label{eq-C2_time_theta_method}
&u_{n+1} = u_n + h (\theta f(t_{n+1}, u_{n+1}) + (1-\theta )f(t_n,u_n)),\\
&u(0) = u_0,
\end{align*}
being $ h $ the time step size and $ t_n = nh $ for $ n=0,...,N $, with $ N=T/h $. Here $ \theta \in [0, 1] $ is a fixed parameter. The $ \theta $-method is considered here as basic method since it represents the most simple Runge-Kutta method (and also linear multistep method). The case of $ \theta=0.5 $ is of second order and is called Crank-Nicolson method. For $ \theta=0 $ we have the so called (explicit) Forward Euler method and for $ \theta=1 $ the (implicit) Backward Euler (BE) method.

\subsection{Implicit-explicit Runge-Kutta schemes}
One of the main goals of this thesis is the construction of efficient solvers for the resolution of the incompressible turbulent Navier-Stokes equations. The solver efficiency can be addressed not only by the use of efficient time integrators, but also by the application of efficient algebraic solvers for the final discrete system of equations. In this direction, the time integration scheme can help to construct smaller systems of equation segregating the different variables that appear on the problem and allowing to solve efficiently each uncoupled variable separately.

In \Chap{SRK} and \Chap{SVMS} we will consider the application of Implicit-Explicit (IMEX) Runge-Kutta methods for the time integration of the Navier-Stokes equations. The aim is to take advantage of the IMEX schemes for Runge-Kutta methods to uncouple the pressure and velocity degrees of freedom when solving these equations. We also want to use the Runge-Kutta background to implement an adaptive time stepping technique to solve efficiently transient incompressible flow problems.

Given an ODE problem of the type
\begin{equation}
\label{eq-C2_time_ODE}
\frac{\partial u}{\partial t}=f(u)+g(u), 
\end{equation}
being $f$ and $g$ different operators which definition depends on the specific problem, an IMEX scheme consists of applying an explicit discretization for the operator $f$ and an implicit discretization for $g$. This approach comes from the fact that ODEs usually are composed by operators of different nature. For instance, thinking in the convection-diffusion problem we have two different operators, which represent the convection term (let us denote it by $f$) and the diffusion term (which we will denote as $g$). As it is well known, the convection term is often nonlinear (i.e. burgers equation) while the diffusion term is generally linear and stiff. When $f\equiv0$, problem (\ref{eq-C2_time_ODE}) results in a stiff and linear system, which is natural to be solved using an implicit scheme. Otherwise, if $g\equiv0$ the problem becomes nonlinear and it could be convenient to be solved using an explicit time integration scheme. Ascher et al. in \cite{ascher_implicit-explicit_1995} study some multistep IMEX methods for convection-diffusion problem type. Often these type of methods are used in conjunction with spectral methods, see \cite{canuto_spectral_1988, kim_application_1985}.

The IMEX approach can be used not only for multistep schemes, but also for Runge-Kutta time integration techniques. The multistage nature of the Runge-Kutta methods also make feasible IMEX schemes with even better properties than multistep methods, see \cite{ascher_implicit-explicit_1997} where some Runge-Kutta IMEX schemes are developed for the convection-diffusion problem.

The idea of the Runge-Kutta methods is to approximate the integral $u(t_{n+1})=u(t_n)+\int_{t_n}^{t_{n+1}}\left[f(u)+g(u)\right]\ dt$ using a numerical quadrature with the points $c_1,...,c_s$ and their weights $b_1,...b_s$, which leads to
\begin{equation}
\label{eq-C2_time_ODE_int}
u(t_{n+1})=u(t_n)+h\sum_{i=1}^sb_i\left(f(u(t_n+c_ih))+g(u(t_n+c_ih))\right)+\mbox{Error}.
\end{equation}

Hereafter we will write $t_i$ instead of $t_n+c_ih$. Suppose we have an approximation $u_n$ to $u(t_n)$); to use (\ref{eq-C2_time_ODE_int}) we also need values $u_i$ to put in for $u(t_i)$. We compute them also by numerical quadratures on the same nodes:
\begin{equation}
\label{eq-C2_time_RK_i}
u_i=u_n+h\sum_{j=1}^sa_{ij}\left(f(u_j)+g(u_j)\right).
\end{equation}
In general this is a set of implicit equations, which we solve and use in (\ref{eq-C2_time_ODE_int}) for our next value
\begin{equation}
\label{eq-C2_time_RK_n+1}
u_{n+1}=u_n+h\sum_{i=1}^sb_i\left(f(u_i)+g(u_i)\right).
\end{equation}

The formulation (\ref{eq-C2_time_RK_i})-(\ref{eq-C2_time_RK_n+1}) define a Runge-Kutta method, which we designate by displaying its coefficient in the called Butcher tableau:
\begin{equation}
\label{eq-C2_time_Butcher_tab}
\begin{array}{c|cccc}
c_1&a_{11}&a_{12}&...&a_{1s}\\
c_2&a_{21}&a_{22}&...&a_{2s}\\
\vdots&\vdots&\vdots&\ddots&\vdots\\
c_s&a_{s1}&a_{s2}&...&a_{ss}\\
\hline
 &b_1&b_2&...&b_s
\end{array}
\end{equation}

% Runge-Kutta for NSI
Runge-Kutta techniques have been widely used for a lot of ODE problem types. The Navier-Stokes semidiscrete problem is not an exception and the use of Runge-Kutta methods for its time integration can be easily found in the literature, e. g. \cite{nikitin_third-order-accurate_2006,sanderse_energy-conserving_2013,sanderse_accuracy_2012,sterner_semi-implicit_1997}. However, like the multistep methods, the Runge-Kutta schemes need to solve several systems of equations at each time step. Further, when we use an implicit scheme all stages could be coupled, resulting a large system of equations to be solved. This drawback can be bypassed using an explicit scheme which only needs to evaluate the operators that arise from the Navier-Stokes problem. But the use of an explicit scheme, as it is well known, involve a restriction in the time-step size in order to ensure stability, see for instance the chapter IV.2 in \cite{hairer_solving_1993}. 

Diagonally Implicit Runge-Kutta methods (DIRK) can be used to avoid stability problems and solving implicitly each Runge-Kutta stage uncoupledly, see \cite{alexander_diagonally_1977}. This technique consists on setting all the Butcher tableau values $a_{ij}$ in (\ref{eq-C2_time_Butcher_tab}) that are above the diagonal to zero. That is, $a_{ij} = 0$ for all $j>i$. In fact, in \cite{alexander_diagonally_1977} the use of the DIRK term is what in \cite{hairer_solving_1993} is referred by Singly Diagonally Implicit Runge-Kutta methods (SDIRK) which means that all the diagonal terms are equal, $a_{ii}=\gamma$. As pointed out by Alexander in \cite{alexander_diagonally_1977}, the use of SDIRK methods allow to use the same LU-factorization when solving repeatedly the multistage system of equations.

% Pressure segregation
An interesting issue when solving the transient Navier-Stokes problem is the decoupling of the velocity and pressure. There are several techniques to deal with this approach that consists on solving separately the velocity degrees of freedom and the pressure by approximating the coupling terms. One of them, for example, is the widely used Fractional step method, \cite{donea_finite_1982}. There also are some works done in this direction for multistep methods, see for instance \cite{kim_application_1985}. Nikitin in \cite{nikitin_third-order-accurate_2006} suggested a Runge-Kutta method which decouples pressure and velocity by using a pressure splitting technique on the last step of the scheme.

% Adaptivee time stepping
Also related with the time integration procedures, there appears the idea of using an adaptive time stepping technique. Adaptive time stepping is an interesting tool that allows to control the accuracy of the time integration, but also improves the simulation efficiency. In this direction, the Runge-Kutta method provides an excellent background to implement this computational tool since we can use the different stages to compute an error estimate at each time step. John et al. in \cite{john_adaptive_2010} studied some time stepping control methods applied to different types of integration schemes, including the DIRK scheme. A more specific step-control analysis for explicit and implicit Runge-Kutta methods is done in \cite{hairer_solving_1993}, where a predictive controller is also proposed. More adaptive time step techniques are proposed in \cite{gresho_adaptive_2008} for convection-diffusion equation and in \cite{kay_adaptive_2010} for the Navier-Stokes equations. Nikitin in \cite{nikitin_third-order-accurate_2006} also includes a section dedicated to the adaptive time step.


\subsection{Stability and order of convergence}
Some definitions of stability and order of convergence need to be introduced since it will be used to charachterize some of the methods used in forthcoming chapters. Deeper explanations on stability and order conditions can be found in \cite{hairer_solving_2008,hairer_solving_1993}.

When analysing stability of ODEs, the solution of the Dahlquist's equation $ u'=\lambda u $ is studied. After applying the implicit Euler method it reads
\begin{equation}
\label{eq-C2_Dahlquist}
u_1=u_0+h\lambda u_1,
\end{equation}
being $ h $ the step size. The solution to \Eq{C2_Dahlquist} is
\begin{equation}
\label{eq-C2_Dahlquist_sol}
u_1=R(h\lambda)u_0,
\end{equation}
where $ R(z) $ is called the stability function for $ z\in\mathbb{C} $. For $ \theta $-methods, the stability function reads $ R(z)=\frac{1+z(1-\theta)}{1-z\theta} $.

Under these definitions we say that a method is \textit{A-stable} if 
\begin{equation}
\label{eq-C2_A_stable}
z\in\mathbb{C}^-,\quad\mbox{with $\quad\mathbb{C^-}:=\{z\in\mathbb{C}|\ Re\ z\le0\} $}.
\end{equation}
It can be shown that a $ \theta $-method is A-stable for all $ \theta\ge\dfrac{1}{2} $, see \cite{lambert_numerical_1991}. Furthermore, we say that a method is \textit{L-stable} when it is A-stable and additionaly it satisfies
\begin{equation}
\label{eq-C2_L_stable}
\lim_{z\rightarrow\infty}R(z)=0.
\end{equation}

Let us consider a Runge-Kutta method given by equations \Eq{C2_time_RK_i}-\Eq{C2_time_RK_n+1}. We say that a Runge-Kutta method has \textit{order p} if 
\begin{equation}
\label{eq-C2_time_RK_order}
\|u(t_{n+1})-u_{n+1}\|\le Ch^{p+1}.
\end{equation}

\section{Conclusions}
\label{sec-C2_conclusions}
In this chapter we have settled, with a certain degree of detail, the formulations that will be used during the forthcoming chapters. If the reader is interested in going further on the description of the concepts stated during the chapter, more information can be found in the references that have been provided along this introductory part.

We have defined the governing equations of the flow motion, which lead to the well-known Navier-Stokes equations for incompressible flows. Some insights are also given in the first section of this chapter related to the influence of the pressure field and the incompressibility constraint in the fluid motion.

Some mathematical notation have been introduced in the second section of this chapter. This notation is a basic tool for the definition of the variational formulation of the Navier-Stokes equations, which has been stated in the third section. We have split the description of the variational formulation in three different subsections, starting with the definition of such formulation at the continuous level. After that, the definition of the FE method is stated, before the development of the semi-discrete problem.

Once we have settled the variational formulation of the Navier-Stokes equations, the Variational Multiscale method has been introduced in the fourth section. A deep description of this method is given focusing on the three different ingredients that define a particular version of such method. More precisely, we have studied the effects of the selecting the subscales to be quasi-static or dynamic, linear or nonlinear and orthogonal or non-orthogonal with respect to the FE space.

Finally, the time integration of the semi-discrete problem has been analyzed in the fifth section of this chapter. We have defined two different groups of time integration schemes. One based on the called \textit{$\theta$-method} and the other based on the \textit{Runge-Kutta} schemes. Basic concepts of stability and order of convergence have been given in this last section.
