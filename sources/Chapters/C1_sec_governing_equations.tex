In this section we will briefly state the basic concepts that one have to know about fluid mechanics and its mathematical description, which will be used in the forthcomming chapters.

Let us start with a review of the equations that govern the fluid flow. In this thesis we will restrict to constant-property Newtonian fluids. 

\subsection{The momentum equation}
\label{subsec-momentum}
From the Newton's second law, the momentum equation relates the fluid particle acceleration, $\frac{D\u}{Dt}$ to the external forces acting on such fluid particle. The external forces can be decomposed into the surface forces and the body forces, $\f$. The surface forces are described by the stress tensor $\boldsigma(\x,t)$ that is symmetric and, for constant-property Newtonian fluids, is defined by
\begin{equation}
\label{eq-stress_tensor_def}
\boldsigma = -\hat{p}\boldI + 2\mu\varepsilon(\u), 
\end{equation}
where $\hat{p}$ is the pressure field, $\boldI$ the identity tensor, $\mu$ the viscosity and $\varepsilon(\u)$ the strain rate tensor, which is defined by the following expression
\begin{equation}
\label{eq-strain_tensor_def}
\varepsilon(\u)=\frac{1}{2}\left[\nabla\u+\left(\nabla\u\right)^T\right].
\end{equation}

The acceleration of the fluid particle is caused by the external forces according to the momentum equation
\begin{equation}
\label{eq-momentum_equation_def1}
\rho\frac{D\u}{Dt}=\nabla\cdot\boldsigma+\rho\f,
\end{equation}
being $\rho$ the density of the flow. Introducing the defintion of the stress tensor \Eq{stress_tensor_def} into \Eq{momentum_equation_def}, and dividing by the density $\rho$, which we consider to be constant, we get an alternative expression of the momentum equation
\begin{equation}
\label{eq-momentum_equation_def2}
\frac{D\u}{Dt}=-\nabla p+\nabla\cdot(2\nu\varepsilon(\u))+\f,
\end{equation}
where we have used that $(1/\rho)\nabla\hat{p}=\nabla(\hat{p}/\rho)=\nabla p$, being $p:=\hat{p}/\rho$ the kinematic pressure, and $\nu=\mu/\rho$ the kinematic viscosity.

\subsection{Mass conservation}
\label{subsec-mass_conservation}
The mass conservation or continuity equation is given by 
\begin{equation}
\label{eq-continuity_equation_def1}
\frac{\partial\rho}{\partial t}+\nabla\cdot(\rho\u)=0,
\end{equation}
which for constant density can be simplified to the kinematic condition that the velocity field be solenoidal or, what is the same, divergence-free:
\begin{equation}
\label{eq-continuity_equation_def2}
\nabla\cdot\u=0.
\end{equation}

\subsection{Navier-Stokes equations}
\label{subsec-NS_equations}
Let $\Omega$ be a bounded domain of $\mathbb{R}^d$, where $d=2,3$ is the number of space dimensions, $\Gamma=\partial\Omega$ its boundary and $(0,T]$ the time interval. The strong form of the steady Navier-Stokes problem that govern the fluid flow motion consists of finding the velocity field $\u$ and the pressure field $p$ such that 
\begin{align}
\label{eq-NS_strong_momentum}
\frac{\partial\u}{\partial t}+\u\cdot\nabla\u-\nu\Delta\u+\nabla p=\f&\quad\mbox{in }\Omega\times(0,T],\\
\label{eq-NS_strong_continuity}
\nabla\cdot\u=0&\quad\mbox{in }\Omega\times(0,T],
\end{align}
where the decomposition of the velocity material derivative into the temporal partial derivative plus the convective derivative, $\frac{D\u}{Dt}=\frac{\partial\u}{\partial t}+\u\cdot\nabla\u$, and the fact that the velocity field is solenoidal have been used to simplify the momentum equation \Eq{momentum_equation_def2} into \Eq{NS_strong_momentum}.

Equations \Eq{NS_strong_momentum} and \Eq{NS_strong_continuity} need to be supplied with apropiate boundary and initial conditions. The boundary $\Gamma$ is divided into the Dirichlet ($\Gamma_D$) and the Neumann ($\Gamma_N$) parts such that $\Gamma_D\cup\Gamma_N=\Gamma$ and $\Gamma_D\cap\Gamma_N=\O$. Then, the boundary and intial conditions can be written as
\begin{align}
\label{eq-NS_strong_Dir}
\u&=\u_g&\mbox{on $\Gamma_D\times(0,T]$,}\\
\label{eq-NS_strong_Neu}
(-p\cdot\mathbf{I}+\nu(\nabla\u+\nabla\u^T))\cdot\mathbf{n}&=\mathbf{t}_N&\mbox{on $\Gamma_N\times(0,T]$,}\\
\label{eq-NS_strong_Ini}
\u(\x,0)&=\u_0(\x)&\mbox{in $\Omega\times\{0\}$,}
\end{align}
$\mathbf{n}$ being the unit outward vector normal to $\Gamma$.

\subsection{Pressure and mass conservation}
\label{subsec-pressure_mass_conservation}

\section{Functional spaces}
