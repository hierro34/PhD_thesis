In this section, our aim is to state some basic concepts about the time discretization. Once we have defined the semi-discrete problem, as it has been formulated in Section \ref{subsec-variational_semidiscrete}, we end up with an Ordinary Differential Equation (ODE), which has to be integrated in order to get the solution of the problem.

In the current work we only will consider the called \textit{direct integration} methods. The \textit{direct integration} of the transient equations rely on a numerical step-by-step procedure, where the word \textit{direct} means that no transformation of the ODE problem is carried out a priori. Looking at the litarature, many techniques can be found based in this kind of procedure. For instance in \cite{Per Klaus-Jürgen Bathe} the description of the following methods can be found: the central difference method, the Houbold method, the Newmark method or the $ \theta $-method. An exhaustive analysis of such methods can be found in \cite{belytscho_hughes}.

Other ... BDF, generalized-alpha, runge-kutta.


\subsection{$ \theta $-metohd}
\subsection{Runge-Kutta schemes}