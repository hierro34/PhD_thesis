% Appendix B

\chapter{VMS methods implementation} % Main appendix title
\label{appendix-VMS_implementation}

In this appendix we give some tips and some comments on how the VMS methods defined in Chapter \ref{chap-Rb_VMS} are implemented in a FE software.

We will describe the algorithms used to implement the dynamic and nonlinear versions for both the ASGS and the OSS methods. The quasi-static alternatives and the linear subscales definition can be though as a particular case of the former version.

\subsection*{Algebraic Subgrid Scale (ASGS)}
The Algebraic Subgrid Scale (ASGS) method is characterized by the projection $\mathcal{P}$ appearing in the right-hand side of \Eq{C2_velo_sgs} and \Eq{C2_press_sgs} defined as
\begin{equation}
\label{4}
\mathcal{P}:=\mathbf{I}
\end{equation}

Then, the equivalent Navier-Stokes problem expressions are given by
\begin{align}
\label{5}
(\partial_t\u_h,\v_h)+(\partial_t\tilde{\u},\v_h)&+B(\a;[\u_h,p_h],[\v_h,q_h])\\\nonumber
&+\left(\tilde{\u},\mathcal{L}_{\a}^*(\v_h,q_h)\right)_h-\left(\tilde{p},\nabla\cdot\v_h\right)=\left<\f,\v_h\right>,
\end{align}
together with the subscales equations
\begin{align}
\label{eq-A2_velo_asgs}
\partial_t\tilde{\u}+\tau_m^{-1}\tilde{\u}=\R_u,\\
\label{eq-A2_press_asgs}
\tau_c^{-1}\tilde{p}=R_p.
\end{align}

The previous expressions are solved using the Picard method for nonlinear equations. Once discretized in time using a Backward Euler discretization, we have that at the step $ n+1 $ and each nonlinear iteration $ i $, the values of $ \mathbf{u}_h^{n+1,i} $ and $ p_h^{n+1,i} $ are calculated solving the following decomposed equations
\begin{align}
\label{7}
&\frac{1}{\delta t}(\mathbf{u}_h^{n+1,i},\mathbf{v}_h)+(\mathbf{a}^i\cdot\nabla\mathbf{u}_h^{n+1,i},\mathbf{v}_h)+\nu(\nabla\mathbf{u}_h^{n+1,i},\nabla\mathbf{v}_h)-(p_h^{n+1,i},\nabla\cdot\mathbf{v}_h)\\\nonumber
&+(q_h,\nabla\cdot\mathbf{u}_h^{n+1,i})-\tau_{m,t}(\mathbf{a}^i\cdot\nabla\mathbf{u}_h^{n+1,i}-\nu\Delta\mathbf{u}_h^{n+1,i}+\nabla p_h^{n+1,i},-\mathbf{a}^i\cdot\nabla\mathbf{v}_h-\nu\Delta\mathbf{v}_h-\nabla q_h)\\\nonumber
&+\tau_c(\nabla\cdot\mathbf{u}_h^{n+1,i},\nabla\cdot\mathbf{v}_h)-\tau_{m,t}\frac{1}{\delta t}(\mathbf{a}^i\cdot\nabla\mathbf{u}_h^{n+1,i}-\nu\Delta\mathbf{u}_h^{n+1,i}+\nabla p_h^{n+1,i},\mathbf{v}_h)\\\nonumber
&-\tau_{m,t}\frac{1}{\delta t}(\mathbf{u}_h^{n+1,i},-\mathbf{a}^i\cdot\nabla\mathbf{v}_h-\nu\Delta\mathbf{v}_h-\nabla q_h)-\tau_{m,t}\frac{1}{\delta t^2}(\mathbf{u}_h^{n+1,i},\mathbf{v}_h)\\\nonumber
&=\left<\mathbf{f},\mathbf{v}_h\right>+\frac{1}{\delta t}(\mathbf{u}_h^n,\mathbf{v}_h)-\tau_{m,t}(\mathbf{f},-\mathbf{a}^i\cdot\nabla\mathbf{v}_h-\nu\Delta\mathbf{v}_h-\nabla q_h)-\tau_{m,t}\frac{1}{\delta t}(\mathbf{f},\mathbf{v}_h)\\\nonumber
&-\tau_{m,t}\frac{1}{\delta t}(\mathbf{u}_h^n,-\mathbf{a}^i\cdot\nabla\mathbf{v}_h-\nu\Delta\mathbf{v}_h-\nabla q_h)-\tau_{m,t}\frac{1}{\delta t^2}(\mathbf{u}_h^n,\mathbf{v}_h)\\\nonumber
&-\tau_{m,t}\frac{1}{\delta t}(\tilde{\mathbf{u}}^n,-\mathbf{a}^i\cdot\nabla\mathbf{v}_h-\nu\Delta\mathbf{v}_h-\nabla q_h)+\tau_{m,t}\frac{1}{\delta t\tau_m}(\tilde{\mathbf{u}}^n,\mathbf{v}_h),
\end{align}
where
\begin{equation}
\label{8}
\tilde{\mathbf{u}}^{n+1,i}=\tau_{m,t}\frac{1}{\delta t}\tilde{\mathbf{u}}^n+\tau_{m,t}\left[\mathbf{f}-(\mathbf{a}^i\cdot\nabla\mathbf{u}_h^{n+1,i}-\nu\Delta\mathbf{u}_h^{n+1,i}+\nabla p_h^{n+1,i})\right].
\end{equation}
With $ \mathbf{a}^i=\mathbf{u}_h^{n+1,i-1}+\tilde{\mathbf{u}}^{n+1,i-1} $ and $ \tau_{m,t}=\left[\frac{1}{\delta t}+\frac{1}{\tau_m}\right]^{-1} $

The implementation of the dynamic subscales has been done only for a linear approximation. It means that the terms involving the laplacian $ \Delta(\cdot) $ are equal to zero. Given this assumption, the simplified expressions of (\ref{7}) and (\ref{8}) are:
\begin{align}
\label{9}
&\frac{1}{\delta t}(\mathbf{u}_h^{n+1,i},\mathbf{v}_h)+(\mathbf{a}^i\cdot\nabla\mathbf{u}_h^{n+1,i},\mathbf{v}_h)+\nu(\nabla\mathbf{u}_h^{n+1,i},\nabla\mathbf{v}_h)-(p_h^{n+1,i},\nabla\cdot\mathbf{v}_h)\\\nonumber
&+(q_h,\nabla\cdot\mathbf{u}_h^{n+1,i})s-\tau_{m,t}(\mathbf{a}^i\cdot\nabla\mathbf{u}_h^{n+1,i}+\nabla p_h^{n+1,i},-\mathbf{a}^i\cdot\nabla\mathbf{v}_h-\nabla q_h)\\\nonumber
&+\tau_c(\nabla\cdot\mathbf{u}_h^{n+1,i},\nabla\cdot\mathbf{v}_h)-\tau_{m,t}\frac{1}{\delta t}(\mathbf{a}^i\cdot\nabla\mathbf{u}_h^{n+1,i}+\nabla p_h^{n+1,i},\mathbf{v}_h)\\\nonumber
&-\tau_{m,t}\frac{1}{\delta t}(\mathbf{u}_h^{n+1,i},-\mathbf{a}^i\cdot\nabla\mathbf{v}_h-\nabla q_h)-\tau_{m,t}\frac{1}{\delta t^2}(\mathbf{u}_h^{n+1,i},\mathbf{v}_h)\\\nonumber
&=\left<\mathbf{f},\mathbf{v}_h\right>+\frac{1}{\delta t}(\mathbf{u}_h^n,\mathbf{v}_h)-\tau_{m,t}(\mathbf{f},-\mathbf{a}^i\cdot\nabla\mathbf{v}_h-\nabla q_h)-\tau_{m,t}\frac{1}{\delta t}(\mathbf{f},\mathbf{v}_h)\\\nonumber
&-\tau_{m,t}\frac{1}{\delta t}(\mathbf{u}_h^n,-\mathbf{a}^i\cdot\nabla\mathbf{v}_h-\nabla q_h)-\tau_{m,t}\frac{1}{\delta t^2}(\mathbf{u}_h^n,\mathbf{v}_h)-\tau_{m,t}\frac{1}{\delta t}(\tilde{\mathbf{u}}^n,-\mathbf{a}^i\cdot\nabla\mathbf{v}_h-\nabla q_h)\\\nonumber
&+\tau_{m,t}\frac{1}{\delta t\tau_m}(\tilde{\mathbf{u}}^n,\mathbf{v}_h),
\end{align}
with
\begin{equation}
\label{10}
\tilde{\mathbf{u}}^{n+1,i}=\tau_{m,t}\frac{1}{\delta t}\tilde{\mathbf{u}}^n+\tau_{m,t}\left[\mathbf{f}-(\mathbf{a}^i\cdot\nabla\mathbf{u}_h^{n+1,i}+\nabla p_h^{n+1,i})\right].
\end{equation}
An alternative equation to (\ref{10}) can be the equivalent to \Eq{C4_velo_sgs_discrete_impl}.

\subsection*{Orthogonal Subscales (OSS)}
The Orthogonal Subscales (OSS) method is characterized by the following projection definition.
\begin{equation}
\label{11}
\mathcal{P}:=\Pi_\tau^\bot=\mathbf{I}-\Pi_\tau
\end{equation}
Then, the equivalent subscales equations to \Eq{A2_velo_asgs}-\Eq{A2_press_asgs} are given by
\begin{align}
\label{eq-A2_velo_oss}
\partial_t\tilde{\u}+\tau_m^{-1}\tilde{\u}=\Pi_\tau^\bot(\R_u),\\
\label{eq-A2_press_oss}
\tau_c^{-1}\tilde{p}=\Pi_\tau^\bot(R_p).
\end{align}

As we have done for the ASGS method, using a Backward Euler time discretization and the Picard method for solving the nonlinearity, at the time step $ n+1 $ and at each iteration $ i $, $ \mathbf{u}_h^{n+1,i} $ and $ p_h^{n+1,i} $ are calculated solving the following decomposed equations
\begin{align}
\label{14}
&\frac{1}{\delta t}(\mathbf{u}_h^{n+1,i},\mathbf{v}_h)+(\mathbf{a}^i\cdot\nabla\mathbf{u}_h^{n+1,i},\mathbf{v}_h)+\nu(\nabla\mathbf{u}_h^{n+1,i},\nabla\mathbf{v}_h)-(p_h^{n+1,i},\nabla\cdot\mathbf{v}_h)\\\nonumber
&+(q_h,\nabla\cdot\mathbf{u}_h^{n+1,i})-\tau_{m,t}(\mathbf{a}^i\cdot\nabla\mathbf{u}_h^{n+1,i}-\nu\Delta\mathbf{u}_h^{n+1,i}+\nabla p_h^{n+1,i},-\mathbf{a}^i\cdot\nabla\mathbf{v}_h-\nu\Delta\mathbf{v}_h-\nabla q_h)\\\nonumber
&+\tau_c(\nabla\cdot\mathbf{u}_h^{n+1,i},\nabla\cdot\mathbf{v}_h)-\tau_{m,t}\frac{1}{\delta t}(\mathbf{a}^i\cdot\nabla\mathbf{u}_h^{n+1,i}-\nu\Delta\mathbf{u}_h^{n+1,i}+\nabla p_h^{n+1,i},\mathbf{v}_h)\\\nonumber
&-\tau_{m,t}\frac{1}{\delta t}(\mathbf{u}_h^{n+1,i},-\mathbf{a}^i\cdot\nabla\mathbf{v}_h-\nu\Delta\mathbf{v}_h-\nabla q_h)-\tau_{m,t}\frac{1}{\delta t^2}(\mathbf{u}_h^{n+1,i},\mathbf{v}_h)\\\nonumber
&=\left<\mathbf{f},\mathbf{v}_h\right>+\frac{1}{\delta t}(\mathbf{u}_h^n,\mathbf{v}_h)-\tau_{m,t}(\mathbf{f},-\mathbf{a}^i\cdot\nabla\mathbf{v}_h-\nu\Delta\mathbf{v}_h-\nabla q_h)-\tau_{m,t}\frac{1}{\delta t}(\mathbf{f},\mathbf{v}_h)\\\nonumber
&-\tau_{m,t}\frac{1}{\delta t}(\mathbf{u}_h^n,-\mathbf{a}^i\cdot\nabla\mathbf{v}_h-\nu\Delta\mathbf{v}_h-\nabla q_h)-\tau_{m,t}\frac{1}{\delta t^2}(\mathbf{u}_h^n,\mathbf{v}_h)\\\nonumber
&+\tau_{m,t}(\Pi_\tau(\mathbf{R}_u^{i-1}),-\mathbf{a}^i\cdot\nabla\mathbf{v}_h-\nu\Delta\mathbf{v}_h-\nabla q_h)+\tau_c(\Pi_\tau(\nabla\cdot\mathbf{u}^{n+1,i-1}),\nabla\cdot\mathbf{v}_h)\\\nonumber
&+\tau_{m,t}\frac{1}{\delta t}(\Pi_\tau(\mathbf{R}_u^{i-1}),\mathbf{v}_h)-\tau_{m,t}\frac{1}{\delta t}(\tilde{\mathbf{u}}^n,-\mathbf{a}^i\cdot\nabla\mathbf{v}_h-\nu\Delta\mathbf{v}_h-\nabla q_h)+\tau_{m,t}\frac{1}{\delta t\tau_m}(\tilde{\mathbf{u}}^n,\mathbf{v}_h),
\end{align}
where
\begin{equation}
\label{15}
\tilde{\mathbf{u}}^{n+1,i}=\tau_{m,t}\frac{1}{\delta t}\tilde{\mathbf{u}}^n+\tau_{m,t}\left[\mathbf{f}-(\mathbf{a}^i\cdot\nabla\mathbf{u}_h^{n+1,i}-\nu\Delta\mathbf{u}_h^{n+1,i}+\nabla p_h^{n+1,i})\right]-\tau_{m,t}\Pi_\tau(\mathbf{R}_u^{i-1}).
\end{equation}
With $ \mathbf{R}_u^i=\mathbf{f}-(\mathbf{a}^i\cdot\nabla\mathbf{u}_h^{n+1,i}-\nu\Delta\mathbf{u}_h^{n+1,i}+\nabla p_h^{n+1,i})$.

As the ASGS method, the implementation of the dynamic subscales has been done only for a linear approximation. Moreover, some of the terms that appear in (\ref{15}) can be neglected, see Chapter \ref{chap-Rb_VMS}. So the resulting expressions are the following.
\begin{align}
\label{16}
%&\frac{1}{\delta t}(\mathbf{u}_h^{n+1,i},\mathbf{v}_h)+(\mathbf{a}^i\cdot\nabla\mathbf{u}_h^{n+1,i},\mathbf{v}_h)+\nu(\nabla\mathbf{u}_h^{n+1,i},\nabla\mathbf{v}_h)-(p_h^{n+1,i},\nabla\cdot\mathbf{v}_h)\\\nonumber
%&+(q_h,\nabla\cdot\mathbf{u}_h^{n+1,i})-\tau_{m,t}(\mathbf{a}^i\cdot\nabla\mathbf{u}_h^{n+1,i}+\nabla p_h^{n+1,i},-\mathbf{a}^i\cdot\nabla\mathbf{v}_h-\nabla q_h)\\\nonumber
%&+\tau_c(\nabla\cdot\mathbf{u}_h^{n+1,i},\nabla\cdot\mathbf{v}_h)-\tau_{m,t}\frac{1}{\delta t}(\mathbf{a}^i\cdot\nabla\mathbf{u}_h^{n+1,i}+\nabla p_h^{n+1,i},\mathbf{v}_h)\\\nonumber
%&-\tau_{m,t}\frac{1}{\delta t}(\mathbf{u}_h^{n+1,i},-\mathbf{a}^i\cdot\nabla\mathbf{v}_h-\nabla q_h)-\tau_{m,t}\frac{1}{\delta t^2}(\mathbf{u}_h^{n+1,i},\mathbf{v}_h)\\\nonumber
%&=\left<\mathbf{f},\mathbf{v}_h\right>+\frac{1}{\delta t}(\mathbf{u}_h^n,\mathbf{v}_h)-\tau_{m,t}(\mathbf{f},-\mathbf{a}^i\cdot\nabla\mathbf{v}_h-\nabla q_h)-\tau_{m,t}\frac{1}{\delta t}(\mathbf{f},\mathbf{v}_h)\\\nonumber
%&-\tau_{m,t}\frac{1}{\delta t}(\mathbf{u}_h^n,-\mathbf{a}^i\cdot\nabla\mathbf{v}_h-\nabla q_h)-\tau_{m,t}\frac{1}{\delta t^2}(\mathbf{u}_h^n,\mathbf{v}_h)+\tau_{m,t}(\Pi_\tau(\mathbf{R}_u^{i-1}),-\mathbf{a}^i\cdot\nabla\mathbf{v}_h-\nabla q_h)\\\nonumber
%&+\tau_c(\Pi_\tau(\nabla\cdot\mathbf{u}^{n+1,i-1}),\nabla\cdot\mathbf{v}_h)+\tau_{m,t}\frac{1}{\delta t}(\Pi_\tau(\mathbf{R}_u^{i-1}),\mathbf{v}_h)-\tau_{m,t}\frac{1}{\delta t}(\tilde{\mathbf{u}}^n,-\mathbf{a}^i\cdot\nabla\mathbf{v}_h-\nabla q_h)\\\nonumber
%&+\tau_{m,t}\frac{1}{\delta t\tau_m}(\tilde{\mathbf{u}}^n,\mathbf{v}_h),
&\frac{1}{\delta t}(\mathbf{u}_h^{n+1,i},\mathbf{v}_h)+(\mathbf{a}^i\cdot\nabla\mathbf{u}_h^{n+1,i},\mathbf{v}_h)+\nu(\nabla\mathbf{u}_h^{n+1,i},\nabla\mathbf{v}_h)-(p_h^{n+1,i},\nabla\cdot\mathbf{v}_h)\\\nonumber
&+(q_h,\nabla\cdot\mathbf{u}_h^{n+1,i})-\tau_{m,t}(\mathbf{a}^i\cdot\nabla\mathbf{u}_h^{n+1,i}+\nabla p_h^{n+1,i},-\mathbf{a}^i\cdot\nabla\mathbf{v}_h-\nabla q_h)\\\nonumber
&-\tau_{m,t}\frac{1}{\delta t}(\mathbf{u}_h^{n+1,i},-\mathbf{a}^i\cdot\nabla\mathbf{v}_h-\nabla q_h)+\tau_c(\nabla\cdot\mathbf{u}_h^{n+1,i},\nabla\cdot\mathbf{v}_h)\\\nonumber
&=\left<\mathbf{f},\mathbf{v}_h\right>+\frac{1}{\delta t}(\mathbf{u}_h^n,\mathbf{v}_h)-\tau_{m,t}(\mathbf{f},-\mathbf{a}^i\cdot\nabla\mathbf{v}_h-\nabla q_h)\\\nonumber
&+\tau_{m,t}(\Pi_\tau(\mathbf{R}_u^{i-1}),-\mathbf{a}^i\cdot\nabla\mathbf{v}_h-\nabla q_h)\\\nonumber
&-\tau_{m,t}\frac{1}{\delta t}(\tilde{\mathbf{u}}^n,-\mathbf{a}^i\cdot\nabla\mathbf{v}_h-\nabla q_h)+\tau_c(\Pi_\tau(\nabla\cdot\mathbf{u}^{n+1,i-1}),\nabla\cdot\mathbf{v}_h), %\\\nonumber
%&+\tau_{m,t}\frac{1}{\delta t\tau_m}(\tilde{\mathbf{u}}^n,\mathbf{v}_h),
\end{align}
with
\begin{equation}
\label{17}
\tilde{\mathbf{u}}^{n+1,i}=\tau_{m,t}\frac{1}{\delta t}\tilde{\mathbf{u}}^n+\tau_{m,t}\left[\mathbf{f}-(\mathbf{a}^i\cdot\nabla\mathbf{u}_h^{n+1,i}+\nabla p_h^{n+1,i})\right]-\tau_{m,t}\Pi_\tau(\mathbf{R}_u^{i-1}).
\end{equation}

\section*{Terms to be implemented}
In the following table (Table \ref{tab-A2_list_terms}) we list the terms needed to be implemented for each version of the VMS methods used in Chapter \ref{chap-Rb_VMS}.

{\renewcommand*\arraystretch{1.5}
\begin{table}
\centering
\caption{List of VMS terms}
\label{tab-A2_list_terms}
\begin{tabular}{ccccc}
\toprule
\multirow{3}{*}{Term}&\multicolumn{4}{c}{Subgrid method}\\ \cline{2-5}
 &\multicolumn{2}{c}{ASGS}&\multicolumn{2}{c}{OSS}\\ \cline{2-5}
 &\multicolumn{1}{c}{QS}&\multicolumn{1}{c}{DYN}&\multicolumn{1}{c}{QS}&\multicolumn{1}{c}{DYN}\\ 
\midrule
\midrule
$ \frac{1}{\delta t}(\u_h^{n+1,i},\v_h) $ & \tickYes & \tickYes & \tickYes & \tickYes \\
$ (\mathbf{a}^i\cdot\nabla\u_h^{n+1,i},\v_h) $ & \tickYes & \tickYes & \tickYes & \tickYes \\
$ \nu(\nabla\u_h^{n+1,i},\nabla\v_h) $ & \tickYes & \tickYes & \tickYes & \tickYes \\
$ (p_h^{n+1,i},\nabla\cdot\v_h) $ & \tickYes & \tickYes & \tickYes & \tickYes \\
$ (q_h,\nabla\cdot\u_h^{n+1,i}) $ & \tickYes & \tickYes & \tickYes & \tickYes \\
$ \tau_{m,t}(\mathbf{a}^i\cdot\nabla\u_h^{n+1,i},\mathbf{a}^i\cdot\nabla\v_h) $ & \tickYes & \tickYes & \tickYes & \tickYes \\
$ \tau_{m,t}(\mathbf{a}^i\cdot\nabla\u_h^{n+1,i},\nabla q_h) $ & \tickYes & \tickYes & \tickYes & \tickYes \\
$ \tau_{m,t}(\nabla p_h^{n+1,i},\mathbf{a}^i\cdot\nabla\v_h) $ & \tickYes & \tickYes & \tickYes & \tickYes \\
$ \tau_{m,t}(\nabla p_h^{n+1,i},\nabla q_h) $ & \tickYes & \tickYes & \tickYes & \tickYes \\
$ \tau_c(\nabla\cdot\mathbf{u}_h^{n+1,i},\nabla\cdot\mathbf{v}_h) $ & \tickYes & \tickYes & \tickYes & \tickYes \\
$ \tau_{m,t}\frac{1}{\delta t}(\mathbf{a}^i\cdot\nabla\u_h^{n+1,i},\v_h) $ & \tickNo & \tickYes & \tickNo & \tickNo \\
$ \tau_{m,t}\frac{1}{\delta t}(\nabla p_h^{n+1,i},\v_h) $ & \tickNo & \tickYes & \tickNo & \tickNo \\
$ \tau_{m,t}\frac{1}{\delta t}(\u_h^{n+1,i},-\mathbf{a}^i\cdot\nabla\v_h) $ & \tickYes & \tickYes & \tickYes & \tickYes \\
$ \tau_{m,t}\frac{1}{\delta t}(\u_h^{n+1,i},-\nabla q_h) $ & \tickYes & \tickYes & \tickYes & \tickYes \\
$ \tau_{m,t}\frac{1}{\delta t^2}(\u_h^{n+1,i},\v_h) $ & \tickNo & \tickYes & \tickNo & \tickNo \\
$ \left<\mathbf{f},\v_h\right> $ & \tickYes & \tickYes & \tickYes & \tickYes \\
$ \frac{1}{\delta t}(\mathbf{u}_h^n,\v_h) $ & \tickYes & \tickYes & \tickYes & \tickYes \\
$ \tau_{m,t}(\mathbf{f},-\mathbf{a}^i\cdot\nabla\v_h) $ & \tickYes & \tickYes & \tickYes & \tickYes \\
$ \tau_{m,t}(\mathbf{f},-\nabla q_h) $ & \tickYes & \tickYes & \tickYes & \tickYes \\
$ \tau_{m,t}\frac{1}{\delta t}(\mathbf{f},\v_h) $ & \tickNo & \tickYes & \tickNo & \tickNo \\
$ \tau_{m,t}\frac{1}{\delta t}(\mathbf{u}_h^n,-\mathbf{a}^i\cdot\nabla\v_h) $ & \tickYes & \tickYes & \tickYes & \tickYes \\
$ \tau_{m,t}\frac{1}{\delta t}(\mathbf{u}_h^n,-\nabla q_h) $ & \tickYes & \tickYes & \tickYes & \tickYes \\
$ \tau_{m,t}\frac{1}{\delta t^2}(\mathbf{u}_h^n,\v_h) $ & \tickNo & \tickYes & \tickNo & \tickNo \\
$ \tau_{m,t}(\Pi_\tau(\mathbf{R}_u^{i-1}),-\mathbf{a}^i\cdot\nabla\v_h) $ & \tickNo & \tickNo & \tickYes & \tickYes \\
$ \tau_{m,t}(\Pi_\tau(\mathbf{R}_u^{i-1}),-\nabla q_h) $ & \tickNo & \tickNo & \tickYes & \tickYes \\
$ \tau_c(\Pi_\tau(\nabla\cdot\mathbf{u}^{n+1,i-1}),\nabla\cdot\v_h) $ & \tickNo & \tickNo & \tickYes & \tickYes \\
%$ \tau_{m,t}\frac{1}{\delta t}(\Pi_\tau(\mathbf{R}^{i-1}),\v_h) $ & \tickNo & \tickNo & \tickNo & \tickNo \\
$ \tau_{m,t}\frac{1}{\delta t}(\tilde{\mathbf{u}}^n,-\mathbf{a}^i\cdot\nabla\v_h) $ & \tickNo & \tickYes & \tickNo & \tickNo \\
$ \tau_{m,t}\frac{1}{\delta t}(\tilde{\mathbf{u}}^n,-\nabla q_h) $ & \tickNo & \tickYes & \tickNo & \tickNo \\
$ \tau_{m,t}\frac{1}{\delta t\tau_m}(\tilde{\mathbf{u}}^n,\v_h) $ & \tickNo & \tickYes & \tickNo & \tickNo \\
$ \tilde{\mathbf{u}}^{n+1,i} $ & \tickNo & \tickYes & \tickNo & \tickYes \\
\bottomrule
\end{tabular}
\end{table}
}
Note that the quasi-static and the dynamic versions of the OSS method, have to compute almost the same terms. The difference between these two versions is the need to compute explicitly $ \tilde{\u}^{n+1,i} $ even for the linear definition of the subscales, and the definition of the parameter $ \tau_{m,t} $, which for the quasi-static version is $ \tau_{m,t}=\tau_m $.


\section*{Algorithm}
In this section, the generic algorithm followed to implement the VMS methods used in Chapter \ref{chap-Rb_VMS} is described. The nonlinear description of the subscales is considered in Algorithm \ref{alg-A2_vms_algorithm}.

\begin{algorithm}
\caption{VMS method algorithm}
\label{alg-A2_vms_algorithm}
Read (or compute) $ \mathbf{u}_h^0 $ and set $ p_h^0=0 $, $ \tilde{\mathbf{u}}^0=\mathbf{0} $.\\
FOR $ n=0,...,N-1 $ DO:\\
$\quad$ Set $ i=0 $\\
$\quad$ Set $ \mathbf{u}_h^{n+1,0}=\mathbf{u}_h^n $, $ p_h^{n+1,0}=p_h^n $, $ \tilde{\mathbf{u}}^{n+1,0}=\tilde{\mathbf{u}}^n $\\
$\quad$ WHILE (not converged) DO:\\
$\quad\quad$ $ i\leftarrow i+1 $\\
$\quad\quad$ Set $ \mathbf{a}^i=\mathbf{u}_h^{n+1,i-1}+\tilde{\mathbf{u}}^{n+1,i-1} $\\
$\quad\quad$ Compute $ \tau_m $, $ \tau_{m,t} $ and $ \tau_c $\\
$\quad\quad$ Compute $ \mathbf{u}_h^{n+1,i} $ and $ p_h^{n+1,i} $ by solving (\ref{9}) or (\ref{16})\\
$\quad\quad$ Update the subscales $ \tilde{\mathbf{u}}^{n+1,i} $ from (\ref{10}) or (\ref{17}), see Algorithm \ref{alg-A2_sgs_algorithm}\\
$\quad\quad$ For OSS, compute the projections $ \Pi_\tau(\mathbf{R}_u^{i}) $ and $ \Pi_\tau(\nabla\cdot\mathbf{u}^{n+1,i}) $\\
$\quad\quad$ Check convergence\\
$\quad$ END\\
$\quad$ Set up the converged values $ \mathbf{u}_h^{n+1}=\mathbf{u}_h^{n+1,i} $, $ p_h^{n+1}=p_h^{n+1,i} $ and $ \tilde{\mathbf{u}}^{n+1}=\tilde{\mathbf{u}}^{n+1,i} $\\
END
\end{algorithm}

In order to compute the velocity subscale $ \tilde{\mathbf{u}}^{n+1,i} $, we must have an explicit implementation of the residual projection. For the global nonlinear iterations, the residual projection is treated implicitly. In Algorithm \ref{alg-A2_sgs_algorithm}, the steps followed to compute the velocity subscale are described.

\begin{algorithm}
\caption{Velocity subscale computation algorithm}
\label{alg-A2_sgs_algorithm}
Compute the residual without the contribution of the subscale on the convective term. That is:\\
$\quad\quad\quad\mathbf{R}_u^{n+1,i}=\mathbf{f}-\left(\frac{1}{\delta t}(\mathbf{u}_h^{n+1,i}-\mathbf{u}_h^{n})+\mathbf{u}_h^{n+1,i}\cdot\nabla\mathbf{u}_h^{n+1,i}+\nabla p_h^{n+1}\right)$\\
If OSS, substract the projection. Here we have the projection $\Pi(\mathbf{R}_u^{n+1,i-1})$ \\
$\quad\quad\quad\mathbf{R}_u^{n+1,i}-\Pi(\mathbf{R}_u^{n+1,i-1})\rightarrow\mathbf{R}_u^{n+1,i}$\\
WHILE (not converged) DO:\\
$\quad$ Compute, and add to the residual, the contribution of the subscale on the convective term:\\ $\quad\quad\quad\mathbf{R}_u^{n+1,i}-(\tilde{\mathbf{u}}^{n+1,i,k-1}\cdot\nabla\mathbf{u}_h^{n+1,i})\rightarrow\hat{\mathbf{R}}_u^{n+1,i,k-1}$\\
$\quad$ Compute $ \tau_m $, $ \tau_{m,t} $ and $ \tau_c $\\
$\quad$ Update the subscale:\\
$\quad\quad\quad\tilde{\mathbf{u}}^{n+1,i,k}=\tau_{m,t}\left(\frac{1}{\delta t}\tilde{\mathbf{u}}^{n}+\hat{\mathbf{R}}_u^{n+1,i,k-1}\right)$\\
$\quad$ Check convergence\\
END
\end{algorithm}