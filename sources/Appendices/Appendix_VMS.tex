% Appendix B

\chapter{VMS methods implementation} % Main appendix title
\label{appendix-VMS_implementation}

\section*{Navier-Stokes weak form:}
Let $ \Omega $ be an open, bounded and polyhedral domain of $ \mathbb{R}^d $, where $ d=2,3 $, $ \Gamma=\partial\Omega $ its boundary and $ [0,T] $ the time interval. The weak form of the Navier-Stokes weak problem consist in finding $ [\mathbf{u},p]\in\mathbf{L}^2(0,T;\mathcal{V}_0)\times L^1(0,T;\mathcal{Q}_0) $ such that

\begin{equation}
\label{1}
(\partial_t\mathbf{u},\mathbf{v})+\nu(\nabla\mathbf{u},\nabla\mathbf{v})+(\mathbf{u}\cdot\nabla\mathbf{u},\mathbf{v})-(p,\nabla\cdot\mathbf{v})+(q,\nabla\cdot\mathbf{u})=\left<\mathbf{f},\mathbf{v}\right>\quad\quad\forall\mathbf{v}\in\mathcal{V}_0,\quad\forall q\in\mathcal{Q}_0.
\end{equation}

\section*{Subgrid Scale approach}
The Subgrid Scale approach, once discretized in space, is given by the following expresions

\begin{equation}
\label{2}
(\partial_t\mathbf{u}_h,\mathbf{v}_h)+(\partial_t\tilde{\mathbf{u}},\mathbf{v}_h)+B(\mathbf{U}_h,\mathbf{V}_h)+\sum_K\int_K\tilde{\mathbf{u}}\cdot\mathcal{L}^*(\mathbf{V}_h)=\left<\mathbf{f},\mathbf{v}_h\right>
\end{equation}

\begin{equation}
\label{3}
\partial_t\tilde{\mathbf{u}}+\tau_1^{-1}\tilde{\mathbf{u}}=\mathcal{P}(\mathbf{R})
\end{equation}

With, $$B(\mathbf{U}_h,\mathbf{V}_h):=\nu(\nabla\mathbf{u},\nabla\mathbf{v})+(\mathbf{u}\cdot\nabla\mathbf{u},\mathbf{v})-(p,\nabla\cdot\mathbf{v})+(q,\nabla\cdot\mathbf{u}).$$

$\mathbf{R}=\mathbf{F}-\mathcal{L}(\mathbf{U}_h)$ is the residual and $\mathcal{P}$ the projection depending on the method.

\subsection*{Algebraic Subgrid Scale (ASGS)}
The Algebraic Subgrid Scale (ASGS) method is characterized by the following projection definiton.

\begin{equation}
\label{4}
\mathcal{P}:=\mathbf{I}
\end{equation}

Then, the equivalent Navier-Stokes problem expressions are given by

\begin{equation}
\label{5}
(\partial_t\mathbf{u}_h,\mathbf{v}_h)+(\partial_t\tilde{\mathbf{u}},\mathbf{v}_h)+B(\mathbf{U}_h,\mathbf{V}_h)+\sum_K\int_K\tilde{\mathbf{u}}\cdot\mathcal{L}^*(\mathbf{V}_h)=\left<\mathbf{f},\mathbf{v}_h\right>.
\end{equation}

\begin{equation}
\label{6}
\partial_t\tilde{\mathbf{u}}+\tau_1^{-1}\tilde{\mathbf{u}}=\mathbf{R}.
\end{equation}

The previous expressions are solved using the Picard method for nonlinear equtions. Once discretized in time using a Backward Euler discretization, on each iteration $ i $, $ \mathbf{u}_h^{n+1,i} $ and $ p_h^{n+1,i} $ are calculed solving the following decomposed equations

\begin{eqnarray}
\label{7}
\frac{1}{\Delta t}(\mathbf{u}_h^{n+1,i},\mathbf{v}_h)+(\mathbf{a}^i\cdot\nabla\mathbf{u}_h^{n+1,i},\mathbf{v}_h)+\nu(\nabla\mathbf{u}_h^{n+1,i},\nabla\mathbf{v}_h)-(p_h^{n+1,i},\nabla\cdot\mathbf{v}_h)+(q_h,\nabla\cdot\mathbf{u}_h^{n+1,i})-\\\nonumber
-\tau_{1,t}(\mathbf{a}^i\cdot\nabla\mathbf{u}_h^{n+1,i}-\nu\Delta\mathbf{u}_h^{n+1,i}+\nabla p_h^{n+1,i},-\mathbf{a}^i\cdot\nabla\mathbf{v}_h-\nu\Delta\mathbf{v}_h-\nabla q_h)+\tau_2(\nabla\cdot\mathbf{u}_h^{n+1,i},\nabla\cdot\mathbf{v}_h)-\\\nonumber
-\tau_{1,t}\frac{1}{\Delta t}(\mathbf{a}^i\cdot\nabla\mathbf{u}_h^{n+1,i}-\nu\Delta\mathbf{u}_h^{n+1,i}+\nabla p_h^{n+1,i},\mathbf{v}_h)-\tau_{1,t}\frac{1}{\Delta t}(\mathbf{u}_h^{n+1,i},-\mathbf{a}^i\cdot\nabla\mathbf{v}_h-\nu\Delta\mathbf{v}_h-\nabla q_h)-\\\nonumber
-\tau_{1,t}\frac{1}{\Delta t^2}(\mathbf{u}_h^{n+1,i},\mathbf{v}_h)=\left<\mathbf{f},\mathbf{v}_h\right>+\frac{1}{\Delta t}(\mathbf{u}_h^n,\mathbf{v}_h)-\tau_{1,t}(\mathbf{f},-\mathbf{a}^i\cdot\nabla\mathbf{v}_h-\nu\Delta\mathbf{v}_h-\nabla q_h)-\tau_{1,t}\frac{1}{\Delta t}(\mathbf{f},\mathbf{v}_h)-\\\nonumber
-\tau_{1,t}\frac{1}{\Delta t}(\mathbf{u}_h^n,-\mathbf{a}^i\cdot\nabla\mathbf{v}_h-\nu\Delta\mathbf{v}_h-\nabla q_h)-\tau_{1,t}\frac{1}{\Delta t^2}(\mathbf{u}_h^n,\mathbf{v}_h)-\tau_{1,t}\frac{1}{\Delta t}(\tilde{\mathbf{u}}^n,-\mathbf{a}^i\cdot\nabla\mathbf{v}_h-\nu\Delta\mathbf{v}_h-\nabla q_h)+\\\nonumber
+\tau_{1,t}\frac{1}{\Delta t\tau_1}(\tilde{\mathbf{u}}^n,\mathbf{v}_h),
\end{eqnarray}

\begin{equation}
\label{8}
\tilde{\mathbf{u}}^{n+1,i}=\tau_{1,t}\frac{1}{\Delta t}\tilde{\mathbf{u}}^n+\tau_{1,t}\left[\mathbf{f}-(\mathbf{a}^i\cdot\nabla\mathbf{u}_h^{n+1,i}-\nu\Delta\mathbf{u}_h^{n+1,i}+\nabla p_h^{n+1,i})\right].
\end{equation}

With $ \mathbf{a}^i=\mathbf{u}_h^{n+1,i-1}+\tilde{\mathbf{u}}^{n+1,i-1} $ and $ \tau_{1,t}=\left[\frac{1}{\Delta t}+\frac{1}{\tau_1}\right]^{-1} $

The implementation of the dynamic subscales has been done only for a linear approximation. It means that the terms involving the laplacian $ \Delta(\cdot) $ are equal to zero. Given this assumption, the symplified expressions of (\ref{7}) and (\ref{8}) are:

\begin{eqnarray}
\label{9}
\frac{1}{\Delta t}(\mathbf{u}_h^{n+1,i},\mathbf{v}_h)+(\mathbf{a}^i\cdot\nabla\mathbf{u}_h^{n+1,i},\mathbf{v}_h)+\nu(\nabla\mathbf{u}_h^{n+1,i},\nabla\mathbf{v}_h)-(p_h^{n+1,i},\nabla\cdot\mathbf{v}_h)+(q_h,\nabla\cdot\mathbf{u}_h^{n+1,i})-\\\nonumber
-\tau_{1,t}(\mathbf{a}^i\cdot\nabla\mathbf{u}_h^{n+1,i}+\nabla p_h^{n+1,i},-\mathbf{a}^i\cdot\nabla\mathbf{v}_h-\nabla q_h)+\tau_2(\nabla\cdot\mathbf{u}_h^{n+1,i},\nabla\cdot\mathbf{v}_h)-\\\nonumber
-\tau_{1,t}\frac{1}{\Delta t}(\mathbf{a}^i\cdot\nabla\mathbf{u}_h^{n+1,i}+\nabla p_h^{n+1,i},\mathbf{v}_h)-\tau_{1,t}\frac{1}{\Delta t}(\mathbf{u}_h^{n+1,i},-\mathbf{a}^i\cdot\nabla\mathbf{v}_h-\nabla q_h)-\\\nonumber
-\tau_{1,t}\frac{1}{\Delta t^2}(\mathbf{u}_h^{n+1,i},\mathbf{v}_h)=\left<\mathbf{f},\mathbf{v}_h\right>+\frac{1}{\Delta t}(\mathbf{u}_h^n,\mathbf{v}_h)-\tau_{1,t}(\mathbf{f},-\mathbf{a}^i\cdot\nabla\mathbf{v}_h-\nabla q_h)-\tau_{1,t}\frac{1}{\Delta t}(\mathbf{f},\mathbf{v}_h)-\\\nonumber
-\tau_{1,t}\frac{1}{\Delta t}(\mathbf{u}_h^n,-\mathbf{a}^i\cdot\nabla\mathbf{v}_h-\nabla q_h)-\tau_{1,t}\frac{1}{\Delta t^2}(\mathbf{u}_h^n,\mathbf{v}_h)-\tau_{1,t}\frac{1}{\Delta t}(\tilde{\mathbf{u}}^n,-\mathbf{a}^i\cdot\nabla\mathbf{v}_h-\nabla q_h)+\tau_{1,t}\frac{1}{\Delta t\tau_1}(\tilde{\mathbf{u}}^n,\mathbf{v}_h),
\end{eqnarray}

\begin{equation}
\label{10}
\tilde{\mathbf{u}}^{n+1,i}=\tau_{1,t}\frac{1}{\Delta t}\tilde{\mathbf{u}}^n+\tau_{1,t}\left[\mathbf{f}-(\mathbf{a}^i\cdot\nabla\mathbf{u}_h^{n+1,i}+\nabla p_h^{n+1,i})\right].
\end{equation}

\subsection*{Orthogonal Subscales (OSS)}
The Orthogonal Subscales (OSS) method is characterized by the following projection definiton.

\begin{equation}
\label{11}
\mathcal{P}:=\Pi_\tau^\bot=\mathbf{I}-\Pi_\tau
\end{equation}

Then, the equivalent Navier-Stokes problem expressions for this method are given by

\begin{equation}
\label{12}
(\partial_t\mathbf{u}_h,\mathbf{v}_h)+(\partial_t\tilde{\mathbf{u}},\mathbf{v}_h)+B(\mathbf{U}_h,\mathbf{V}_h)+\sum_K\int_K\tilde{\mathbf{u}}\cdot\mathcal{L}^*(\mathbf{V}_h)=\left<\mathbf{f},\mathbf{v}_h\right>.
\end{equation}

\begin{equation}
\label{13}
\partial_t\tilde{\mathbf{u}}+\tau_1^{-1}\tilde{\mathbf{u}}=\Pi_\tau^\bot(\mathbf{R})
\end{equation}

Using a Backward Euler time discretization and the Picard method for solving the nonlinearity, on each iteration $ i $, $ \mathbf{u}_h^{n+1,i} $ and $ p_h^{n+1,i} $ are calculed solving the following decomposed equations

\begin{eqnarray}
\label{14}
\frac{1}{\Delta t}(\mathbf{u}_h^{n+1,i},\mathbf{v}_h)+(\mathbf{a}^i\cdot\nabla\mathbf{u}_h^{n+1,i},\mathbf{v}_h)+\nu(\nabla\mathbf{u}_h^{n+1,i},\nabla\mathbf{v}_h)-(p_h^{n+1,i},\nabla\cdot\mathbf{v}_h)+(q_h,\nabla\cdot\mathbf{u}_h^{n+1,i})-\\\nonumber
-\tau_{1,t}(\mathbf{a}^i\cdot\nabla\mathbf{u}_h^{n+1,i}-\nu\Delta\mathbf{u}_h^{n+1,i}+\nabla p_h^{n+1,i},-\mathbf{a}^i\cdot\nabla\mathbf{v}_h-\nu\Delta\mathbf{v}_h-\nabla q_h)+\tau_2(\nabla\cdot\mathbf{u}_h^{n+1,i},\nabla\cdot\mathbf{v}_h)-\\\nonumber
-\tau_{1,t}\frac{1}{\Delta t}(\mathbf{a}^i\cdot\nabla\mathbf{u}_h^{n+1,i}-\nu\Delta\mathbf{u}_h^{n+1,i}+\nabla p_h^{n+1,i},\mathbf{v}_h)-\tau_{1,t}\frac{1}{\Delta t}(\mathbf{u}_h^{n+1,i},-\mathbf{a}^i\cdot\nabla\mathbf{v}_h-\nu\Delta\mathbf{v}_h-\nabla q_h)-\\\nonumber
-\tau_{1,t}\frac{1}{\Delta t^2}(\mathbf{u}_h^{n+1,i},\mathbf{v}_h)=\left<\mathbf{f},\mathbf{v}_h\right>+\frac{1}{\Delta t}(\mathbf{u}_h^n,\mathbf{v}_h)-\tau_{1,t}(\mathbf{f},-\mathbf{a}^i\cdot\nabla\mathbf{v}_h-\nu\Delta\mathbf{v}_h-\nabla q_h)-\tau_{1,t}\frac{1}{\Delta t}(\mathbf{f},\mathbf{v}_h)-\\\nonumber
-\tau_{1,t}\frac{1}{\Delta t}(\mathbf{u}_h^n,-\mathbf{a}^i\cdot\nabla\mathbf{v}_h-\nu\Delta\mathbf{v}_h-\nabla q_h)-\tau_{1,t}\frac{1}{\Delta t^2}(\mathbf{u}_h^n,\mathbf{v}_h)+\tau_{1,t}(\Pi_\tau(\mathbf{R}^{i-1}),-\mathbf{a}^i\cdot\nabla\mathbf{v}_h-\nu\Delta\mathbf{v}_h-\nabla q_h)+\\\nonumber
+\tau_2(\Pi_\tau(\nabla\cdot\mathbf{u}^{n+1,i-1}),\nabla\cdot\mathbf{v}_h)+\tau_{1,t}\frac{1}{\Delta t}(\Pi_\tau(\mathbf{R}^{i-1}),\mathbf{v}_h)-\tau_{1,t}\frac{1}{\Delta t}(\tilde{\mathbf{u}}^n,-\mathbf{a}^i\cdot\nabla\mathbf{v}_h-\nu\Delta\mathbf{v}_h-\nabla q_h)+\\\nonumber
+\tau_{1,t}\frac{1}{\Delta t\tau_1}(\tilde{\mathbf{u}}^n,\mathbf{v}_h),
\end{eqnarray}

\begin{equation}
\label{15}
\tilde{\mathbf{u}}^{n+1,i}=\tau_{1,t}\frac{1}{\Delta t}\tilde{\mathbf{u}}^n+\tau_{1,t}\left[\mathbf{f}-(\mathbf{a}^i\cdot\nabla\mathbf{u}_h^{n+1,i}-\nu\Delta\mathbf{u}_h^{n+1,i}+\nabla p_h^{n+1,i})\right]-\tau_{1,t}\Pi_\tau(\mathbf{R}^{i-1}).
\end{equation}

With $ \mathbf{R}^i=\mathbf{f}-(\mathbf{a}^i\cdot\nabla\mathbf{u}_h^{n+1,i}-\nu\Delta\mathbf{u}_h^{n+1,i}+\nabla p_h^{n+1,i})$.

As the ASGS method. the implementation of the dynamic subscales has been done only for a linear approximation. So the resulting expressions are the following.

\begin{eqnarray}
\label{16}
\frac{1}{\Delta t}(\mathbf{u}_h^{n+1,i},\mathbf{v}_h)+(\mathbf{a}^i\cdot\nabla\mathbf{u}_h^{n+1,i},\mathbf{v}_h)+\nu(\nabla\mathbf{u}_h^{n+1,i},\nabla\mathbf{v}_h)-(p_h^{n+1,i},\nabla\cdot\mathbf{v}_h)+(q_h,\nabla\cdot\mathbf{u}_h^{n+1,i})-\\\nonumber
-\tau_{1,t}(\mathbf{a}^i\cdot\nabla\mathbf{u}_h^{n+1,i}+\nabla p_h^{n+1,i},-\mathbf{a}^i\cdot\nabla\mathbf{v}_h-\nabla q_h)+\tau_2(\nabla\cdot\mathbf{u}_h^{n+1,i},\nabla\cdot\mathbf{v}_h)-\\\nonumber
-\tau_{1,t}\frac{1}{\Delta t}(\mathbf{a}^i\cdot\nabla\mathbf{u}_h^{n+1,i}+\nabla p_h^{n+1,i},\mathbf{v}_h)-\tau_{1,t}\frac{1}{\Delta t}(\mathbf{u}_h^{n+1,i},-\mathbf{a}^i\cdot\nabla\mathbf{v}_h-\nabla q_h)-\\\nonumber
-\tau_{1,t}\frac{1}{\Delta t^2}(\mathbf{u}_h^{n+1,i},\mathbf{v}_h)=\left<\mathbf{f},\mathbf{v}_h\right>+\frac{1}{\Delta t}(\mathbf{u}_h^n,\mathbf{v}_h)-\tau_{1,t}(\mathbf{f},-\mathbf{a}^i\cdot\nabla\mathbf{v}_h-\nabla q_h)-\tau_{1,t}\frac{1}{\Delta t}(\mathbf{f},\mathbf{v}_h)-\\\nonumber
-\tau_{1,t}\frac{1}{\Delta t}(\mathbf{u}_h^n,-\mathbf{a}^i\cdot\nabla\mathbf{v}_h-\nabla q_h)-\tau_{1,t}\frac{1}{\Delta t^2}(\mathbf{u}_h^n,\mathbf{v}_h)+\tau_{1,t}(\Pi_\tau(\mathbf{R}^{i-1}),-\mathbf{a}^i\cdot\nabla\mathbf{v}_h-\nabla q_h)+\\\nonumber
+\tau_2(\Pi_\tau(\nabla\cdot\mathbf{u}^{n+1,i-1}),\nabla\cdot\mathbf{v}_h)+\tau_{1,t}\frac{1}{\Delta t}(\Pi_\tau(\mathbf{R}^{i-1}),\mathbf{v}_h)-\tau_{1,t}\frac{1}{\Delta t}(\tilde{\mathbf{u}}^n,-\mathbf{a}^i\cdot\nabla\mathbf{v}_h-\nabla q_h)+\tau_{1,t}\frac{1}{\Delta t\tau_1}(\tilde{\mathbf{u}}^n,\mathbf{v}_h),
\end{eqnarray}

\begin{equation}
\label{17}
\tilde{\mathbf{u}}^{n+1,i}=\tau_{1,t}\frac{1}{\Delta t}\tilde{\mathbf{u}}^n+\tau_{1,t}\left[\mathbf{f}-(\mathbf{a}^i\cdot\nabla\mathbf{u}_h^{n+1,i}+\nabla p_h^{n+1,i})\right]-\tau_{1,t}\Pi_\tau(\mathbf{R}^{i-1}).
\end{equation}


\section*{Algorithm}
\begin{center}
\fbox{
\begin{tabular}{l}
Read (or compute) $ \mathbf{u}_h^0 $ and set $ p_h^0=0 $, $ \tilde{\mathbf{u}}^0=\mathbf{0} $.\\
FOR $ n=0,...,N-1 $ DO:\\
$\quad$ Set $ i=0 $\\
$\quad$ Set $ \mathbf{u}_h^{n+1,0}=\mathbf{u}_h^n $, $ p_h^{n+1,0}=p_h^n $, $ \tilde{\mathbf{u}}^{n+1,0}=\tilde{\mathbf{u}}^n $\\
$\quad$ WHILE (not converged) DO:\\
$\quad\quad$ $ i\leftarrow i+1 $\\
$\quad\quad$ Set $ \mathbf{a}^i=\mathbf{u}_h^{n+1,i-1}+\tilde{\mathbf{u}}^{n+1,i-1} $\\
$\quad\quad$ Compute $ \tau_1 $, $ \tau_{1,t} $ and $ \tau_2 $\\
$\quad\quad$ Compute $ \mathbf{u}_h^{n+1,i} $ and $ p_h^{n+1,i} $ by solving (\ref{9}) or (\ref{16})\\
$\quad\quad$ Update the subscales $ \tilde{\mathbf{u}}^{n+1,i} $ from (\ref{10}) or (\ref{17})\\
$\quad\quad$ For OSS, compute the projections $ \Pi_\tau(\mathbf{R}^{i}) $ and $ \Pi_\tau(\nabla\cdot\mathbf{u}^{n+1,i}) $\\
$\quad\quad$ Check convergence\\
$\quad$ END\\
$\quad$ Set up the converged values $ \mathbf{u}_h^{n+1}=\mathbf{u}_h^{n+1,i} $, $ p_h^{n+1}=p_h^{n+1,i} $ and $ \tilde{\mathbf{u}}^{n+1}=\tilde{\mathbf{u}}^{n+1,i} $\\
END
\end{tabular}}
\end{center}

\section*{Subroutines in FEMUSS}

\renewcommand*\arraystretch{1.5}

\begin{center}
\rowcolors{1}{tableShade}{white}
\begin{tabular}{|c|c|c|cccc|}
\hline
\hiderowcolors
\multirow{3}{*}{Id.}&\multirow{3}{*}{Term}&\multirow{3}{*}{Subroutine}&\multicolumn{4}{|c|}{Subgrid method}\\ \cline{4-7}
 & & &\multicolumn{2}{|c|}{ASGS}&\multicolumn{2}{|c|}{OSS}\\ \cline{4-7}
 & & &\multicolumn{1}{|c|}{QS}&\multicolumn{1}{|c|}{DYN}&\multicolumn{1}{|c|}{QS}&\multicolumn{1}{|c|}{DYN}\\ 
\hline
\showrowcolors 
1&$ \frac{1}{\Delta t}(\mathbf{\phi}_j,\mathbf{\phi}_i) $& nsm\_elmbuv & \tickYes & \tickYes & \tickYes & \tickYes \\
2&$ (\mathbf{a}^i\cdot\nabla\mathbf{\phi}_j,\mathbf{\phi}_i) $& nsm\_elmbuv & \tickYes & \tickYes & \tickYes & \tickYes \\
3&$ \nu(\nabla\mathbf{\phi}_j,\nabla\mathbf{\phi}_i) $& nsm\_elmvis & \tickYes & \tickYes & \tickYes & \tickYes \\
4&$ (\psi_j,\nabla\cdot\mathbf{\phi}_i) $& nsm\_elmbpv & \tickYes & \tickYes & \tickYes & \tickYes \\
5&$ (\psi_i,\nabla\cdot\mathbf{\phi}_j) $& nsm\_elmbuq & \tickYes & \tickYes & \tickYes & \tickYes \\
6&$ \tau_{1,t}(\mathbf{a}^i\cdot\nabla\mathbf{\phi}_j,\mathbf{a}^i\cdot\nabla\mathbf{\phi}_i) $& nsm\_elmbuv & \tickYes & \tickYes & \tickYes & \tickYes \\
7&$ \tau_{1,t}(\mathbf{a}^i\cdot\nabla\mathbf{\phi}_j,\nabla\psi_i) $& nsm\_elmbuq & \tickYes & \tickYes & \tickYes & \tickYes \\
8&$ \tau_{1,t}(\nabla\psi_j,\mathbf{a}^i\cdot\nabla\mathbf{\phi}_i) $& nsm\_elmbpv & \tickYes & \tickYes & \tickYes & \tickYes \\
9&$ \tau_{1,t}(\nabla\psi_j,\nabla\psi_i) $& nsm\_elmbpq & \tickYes & \tickYes & \tickYes & \tickYes \\
10&$ \tau_2(\nabla\cdot\mathbf{u}_h^{n+1,i},\nabla\cdot\mathbf{v}_h) $& nsm\_elmdiv & \tickYes & \tickYes & \tickYes & \tickYes \\
11&$ \tau_{1,t}\frac{1}{\Delta t}(\mathbf{a}^i\cdot\nabla\mathbf{\phi}_j,\mathbf{\phi}_i) $& nsm\_elmbuv\_dss & \tickNo & \tickYes & \tickNo & \tickYes \\
12&$ \tau_{1,t}\frac{1}{\Delta t}(\nabla \psi_j,\mathbf{\phi}_i) $& nsm\_elmbpv\_dss & \tickNo & \tickYes & \tickNo & \tickYes \\
13&$ \tau_{1,t}\frac{1}{\Delta t}(\mathbf{\phi}_j,-\mathbf{a}^i\cdot\nabla\mathbf{\phi}_i) $& nsm\_elmbuv & \tickYes & \tickYes & \tickYes & \tickYes \\
14&$ \tau_{1,t}\frac{1}{\Delta t}(\mathbf{\phi}_j,-\nabla \psi_i) $& nsm\_elmbuq & \tickYes & \tickYes & \tickYes & \tickYes \\
15&$ \tau_{1,t}\frac{1}{\Delta t^2}(\mathbf{\phi}_j,\mathbf{\phi}_i) $& nsm\_elmbuv\_dss & \tickNo & \tickYes & \tickNo & \tickYes \\
16&$ \left<\mathbf{f},\mathbf{\phi}_i\right> $& nsm\_elmrhu & \tickYes & \tickYes & \tickYes & \tickYes \\
17&$ \frac{1}{\Delta t}(\mathbf{u}_h^n,\mathbf{\phi}_i) $& nsm\_elmrhu & \tickYes & \tickYes & \tickYes & \tickYes \\
18&$ \tau_{1,t}(\mathbf{f},-\mathbf{a}^i\cdot\nabla\mathbf{\phi}_i) $& nsm\_elmrhu & \tickYes & \tickYes & \tickYes & \tickYes \\
19&$ \tau_{1,t}(\mathbf{f},-\nabla \psi_i) $& nsm\_elmrhp & \tickYes & \tickYes & \tickYes & \tickYes \\
20&$ \tau_{1,t}\frac{1}{\Delta t}(\mathbf{f},\mathbf{\phi}_i) $& nsm\_elmrhu\_dss & \tickNo & \tickYes & \tickNo & \tickYes \\
21&$ \tau_{1,t}\frac{1}{\Delta t}(\mathbf{u}_h^n,-\mathbf{a}^i\cdot\nabla\mathbf{\phi}_i) $& nsm\_elmrhu & \tickYes & \tickYes & \tickYes & \tickYes \\
22&$ \tau_{1,t}\frac{1}{\Delta t}(\mathbf{u}_h^n,-\nabla \psi_i) $& nsm\_elmrhp & \tickYes & \tickYes & \tickYes & \tickYes \\
23&$ \tau_{1,t}\frac{1}{\Delta t^2}(\mathbf{u}_h^n,\mathbf{\phi}_i) $& nsm\_elmrhu\_dss & \tickNo & \tickYes & \tickNo & \tickYes \\
24&$ \tau_{1,t}(\Pi_\tau(\mathbf{R}^{i-1}),-\mathbf{a}^i\cdot\nabla\mathbf{\phi}_i) $& nsm\_elmrhu\_oss & \tickNo & \tickNo & \tickYes & \tickYes \\
25&$ \tau_{1,t}(\Pi_\tau(\mathbf{R}^{i-1}),-\nabla \psi_i) $& nsm\_elmrhp\_oss & \tickNo & \tickNo & \tickYes & \tickYes \\
26&$ \tau_2(\Pi_\tau(\nabla\cdot\mathbf{u}^{n+1,i-1}),\nabla\cdot\mathbf{\phi}_i) $& nsm\_elmrhu\_oss & \tickNo & \tickNo & \tickYes & \tickYes \\
27&$ \tau_{1,t}\frac{1}{\Delta t}(\Pi_\tau(\mathbf{R}^{i-1}),\mathbf{\phi}_i) $& nsm\_elmrhu\_dss & \tickNo & \tickNo & \tickNo & \tickYes \\
28&$ \tau_{1,t}\frac{1}{\Delta t}(\tilde{\mathbf{u}}^n,-\mathbf{a}^i\cdot\nabla\mathbf{\phi}_i) $& nsm\_elmrhu\_dss & \tickNo & \tickYes & \tickNo & \tickYes \\
29&$ \tau_{1,t}\frac{1}{\Delta t}(\tilde{\mathbf{u}}^n,-\nabla \psi_i) $& nsm\_elmrhp\_dss & \tickNo & \tickYes & \tickNo & \tickYes \\
30&$ \tau_{1,t}\frac{1}{\Delta t\tau_1}(\tilde{\mathbf{u}}^n,\mathbf{\phi}_i) $& nsm\_elmrhu\_dss & \tickNo & \tickYes & \tickNo & \tickYes \\
31&$ \tilde{\mathbf{u}}^{n+1,i} $& nsm\_elmsgs\_dss & \tickNo & \tickYes & \tickNo & \tickYes \\
\hline
\end{tabular}
\end{center}

\subsection*{Subroutines diagram}

The following diagrams show where are appearing the diferent subroutines mentioned on the previous table.

%\begin{figure}[h!]
%  \centering
%    \includegraphics[width=0.8\textwidth]{Figures/Diagrama_nsm_matrix}
%  \caption{\textit{nsm\_matrix} subroutine diagram}
%  \label{nsm_matrix}
%\end{figure}
%\begin{figure}[h!]
%  \centering
%    \includegraphics[width=0.8\textwidth]{Figures/Diagrama_nsi_endite}
%  \caption{\textit{nsi\_endite} subroutine diagram}
%  \label{nsi_endite}
%\end{figure}

-------------------------------------------------------


Important considerations on the implementation of the Dynamic Subgrid Scales (Orthogonal or Algebraic) formulation on FEMPAR:
\begin{itemize}
\item[1.] Temporal integration: Backward Euler.
\item[2.] Non-linear tracking of the subscales. Picard scheme.
\item[3.] Picard scheme for the global non-linear iterations.
\item[4.] Linear interpolation.
\item[5.] Lumped matrix for the OSS projection calculation.
\end{itemize}

Keeping in mind the previous considerations, the fully discrete variational problem can be written as (see \textit{R. Codina et al. 2007} eq. 22 with $\theta=1$): given $\mathbf{u}_h^n$ and $\tilde{\mathbf{u}}^n$, find $\mathbf{u}_h^{n+1}$, $p_h^{n+1}$ and $\tilde{\mathbf{u}}^{n+1}$ such that

\begin{align}
\label{1}
&\frac{1}{\Delta t}(\mathbf{u}_h^{n+1},\mathbf{v}_h)+\nu(\nabla\mathbf{u}_h^{n+1},\nabla\mathbf{v}_h)+b((\mathbf{u}_h^{n+1}+\tilde{\mathbf{u}}^{n+1}),\mathbf{u}_h^{n+1},\mathbf{v}_h)-(p_h^{n+1},\nabla\cdot\mathbf{v}_h)+\\\nonumber
&+(q_h,\nabla\cdot\mathbf{u}_h^{n+1})+\frac{1}{\Delta t}(\tilde{\mathbf{u}}^{n+1},\mathbf{v}_h)-\sum_K\langle\tilde{\mathbf{u}}^{n+1},N((\mathbf{u}_h^{n+1}+\tilde{\mathbf{u}}^{n+1}),\mathbf{v}_h)+\nabla q_h\rangle+\\\nonumber
&+\sum_K(\tau_2\nabla\cdot\mathbf{u}_h^{n+1},\nabla\cdot\mathbf{v}_h)=\langle\mathbf{v}_h,\mathbf{f}\rangle+\frac{1}{\Delta t}(\mathbf{u}_h^{n},\mathbf{v}_h)+\frac{1}{\Delta t}(\tilde{\mathbf{u}}^{n},\mathbf{v}_h),\\
\label{2}
&\tilde{\mathbf{u}}^{n+1}=\tau_t\left(\frac{1}{\Delta t}\tilde{\mathbf{u}}^{n}+\mathcal{P}^\perp(\mathbf{R})\right).
\end{align}

Where $b(\mathbf{u},\mathbf{v},\mathbf{w})=(\mathbf{u}\cdot\nabla\mathbf{v},\mathbf{w})$, $N(\mathbf{u},\mathbf{v})=\mathbf{u}\cdot\nabla\mathbf{v}$ and $$\mathbf{R}=\mathbf{f}-\left(\frac{1}{\Delta t}(\mathbf{u}_h^{n+1}-\mathbf{u}_h^{n})+N((\mathbf{u}^{n+1}+\tilde{\mathbf{u}}^{n+1}),\mathbf{u}_h^{n+1})+\nabla p_h^{n+1}\right).$$

Integrating by parts the pressure term, using a Picard linearization scheme and introducing (\ref{2}) into (\ref{1}), the $i$-th non-linear iteration consists on solving:

\begin{footnotesize}
\begin{align}
\label{3}
&\frac{1}{\Delta t}(\mathbf{u}_h^{n+1,i},\mathbf{v}_h)+\nu(\nabla\mathbf{u}_h^{n+1,i},\nabla\mathbf{v}_h)+b(\mathbf{a}^{n+1,i-1},\mathbf{u}_h^{n+1,i},\mathbf{v}_h)-(p_h^{n+1},\nabla\cdot\mathbf{v}_h)+(q_h,\nabla\cdot\mathbf{u}_h^{n+1,i})-\\\nonumber
&-\frac{1}{\Delta t}\left(\tau_t\left[\frac{1}{\Delta t}\mathbf{u}_h^{n+1,i}+N(\mathbf{a}^{n+1,i-1},\mathbf{u}_h^{n+1,i})+\nabla p_h^{n+1}\right],\mathbf{v}_h\right)+\\\nonumber
&+\left(\tau_t\left[\frac{1}{\Delta t}\mathbf{u}_h^{n+1,i}+N(\mathbf{a}^{n+1,i-1},\mathbf{u}_h^{n+1,i})+\nabla p_h^{n+1}\right],N(\mathbf{a}^{n+1,i-1},\mathbf{v}_h)+\nabla q_h\right)+\\\nonumber
&+\tau_2(\nabla\cdot\mathbf{u}_h^{n+1,i},\nabla\cdot\mathbf{v}_h)=\\\nonumber
&=\langle\mathbf{v}_h,\mathbf{f}\rangle+\frac{1}{\Delta t}(\mathbf{u}_h^{n},\mathbf{v}_h)+\frac{1}{\Delta t}(\tilde{\mathbf{u}}^{n},\mathbf{v}_h)-\\\nonumber
&-\frac{1}{\Delta t}\left(\tau_t\left[\frac{1}{\Delta t}\mathbf{u}_h^{n+1,i}+\mathbf{f}+\frac{1}{\Delta t}\tilde{\mathbf{u}}^n\right],\mathbf{v}_h\right)+\\\nonumber
& +\left(\tau_t\left[\frac{1}{\Delta t}\mathbf{u}_h^{n}+\mathbf{f}+\frac{1}{\Delta t}\tilde{\mathbf{u}}^n\right],N(\mathbf{a}^{n+1,i-1},\mathbf{v}_h)+\nabla q_h\right)+\tau_2(\nabla\cdot\mathbf{u}_h^{n},\nabla\cdot\mathbf{v}_h).\\\nonumber
\end{align}
\end{footnotesize}

With $\mathbf{a}^{n+1,i-1}=\mathbf{u}_h^{n+1,i-1}+\tilde{\mathbf{u}}^{n+1,i-1}$.

The correspondences between (\ref{1}) and (\ref{3}) are:
\begin{itemize}
\item[Line 1:] LHS Galerkin terms plus the contribution of $\lbrace b(\tilde{\mathbf{u}}^{n+1},\mathbf{u}_h^{n+1},\mathbf{v}_h)\rbrace$.
\item[Line 2:] LHS part when introducing (\ref{2}) into $\lbrace\frac{1}{\Delta t}(\tilde{\mathbf{u}}^{n+1},\mathbf{v}_h)\rbrace$.
\item[Line 3:] LHS part when introducing (\ref{2}) into $\lbrace-\sum_K\langle\tilde{\mathbf{u}}^{n+1},N((\mathbf{u}_h^{n+1}+\tilde{\mathbf{u}}^{n+1}),\mathbf{v}_h)+\nabla q_h\rangle\rbrace$.
\item[Line 4:] Divergence stabilization term.
\item[Line 5:] Galerkin RHS plus the contribution of the subscale temporal derivative.
\item[Line 6:] RHS part when introducing (\ref{2}) into $\lbrace\frac{1}{\Delta t}(\tilde{\mathbf{u}}^{n+1},\mathbf{v}_h)\rbrace$.
\item[Line 7:] RHS part when introducing (\ref{2}) into $\lbrace-\sum_K\langle\tilde{\mathbf{u}}^{n+1},N((\mathbf{u}_h^{n+1}+\tilde{\mathbf{u}}^{n+1}),\mathbf{v}_h)+\nabla q_h\rangle\rbrace$.
\end{itemize}

For OSS, we must add the following terms on the RHS:
$$-\tau_t\left(\Pi(\mathbf{R}_{oss}^{n+1,i-1}),N(\mathbf{a}^{n+1,i-1},\mathbf{v}_h)+\nabla q_h\right)+\frac{\tau_t}{\Delta t}\left(\Pi(\mathbf{R}_{oss}^{n+1,i-1}),\mathbf{v}_h\right)+\tau_2\left(\Pi(\nabla\cdot\mathbf{u}_h^{n+1,i-1}),\nabla\cdot\mathbf{v}\right),$$

with the residual calculated with the velocity and the subscale on the previous nonlinear iteration ($\mathbf{u}_h^{n+1,i}$ and $\tilde{\mathbf{u}}^{n+1,i-1}$): $$\mathbf{R}_{oss}^{n+1,i-1}=\mathbf{f}-\left(\frac{1}{\Delta t}(\mathbf{u}_h^{n+1,i-1}-\mathbf{u}_h^{n})+N((\mathbf{u}^{n+1,i-1}+\tilde{\mathbf{u}}^{n+1,i-1}),\mathbf{u}_h^{n+1})+\nabla p_h^{n+1}\right).$$

The update of the subscales is done iteratively as follows:

%\begin{itemize}
%\item[1.] Compute the residual without the contribution of the subscale on the convective term. That is
%$$\mathbf{R}^{n+1,i}=\mathbf{f}-\left(\frac{1}{\Delta t}(\mathbf{u}_h^{n+1,i}-\mathbf{u}_h^{n})+N((\mathbf{u}_h^{n+1,i},\mathbf{u}_h^{n+1,i})+\nabla p_h^{n+1}\right).$$
%\item[2.] If OSS, substract the projection.
%\item[3.] Iterate until converged. On the $k$-th iteration we have:
%\item[3.1] Compute, and add to the residual, the contribution of the subscale on the convective term: $$\mathbf{R}^{n+1,i}-(\tilde{\mathbf{u}}^{n+1,i,k-1}\cdot\nabla\mathbf{u}_h^{n+1,i})\rightarrow\hat{\mathbf{R}}^{n+1,i,k-1}$$
%\item[3.2] Compute stabilization parameters.
%\item[3.3] Update the subscale: $$\tilde{\mathbf{u}}^{n+1,i,k}=\tau_t\left(\frac{1}{\Delta t}\tilde{\mathbf{u}}^{n}+\mathcal{P}^\perp(\hat{\mathbf{R}}^{n+1,i,k-1})\right)$$ 
%\item[3.4] Check if converged.
%\end{itemize}
\begin{center}
\fbox{
\begin{tabular}{l}
Compute the residual without the contribution of the subscale on the convective term. That is:\\
$\quad\quad\quad\mathbf{R}^{n+1,i}=\mathbf{f}-\left(\frac{1}{\Delta t}(\mathbf{u}_h^{n+1,i}-\mathbf{u}_h^{n})+N((\mathbf{u}_h^{n+1,i},\mathbf{u}_h^{n+1,i})+\nabla p_h^{n+1}\right)$\\
If OSS, substract the projection. Here we have the projection $\Pi(\mathbf{R}_{oss}^{n+1,i-1})$ \\
$\quad\quad\quad\mathbf{R}^{n+1,i}-\Pi(\mathbf{R}_{oss}^{n+1,i-1})\rightarrow\mathbf{R}^{n+1,i}$\\
WHILE (not converged) DO:\\
$\quad$ Compute, and add to the residual, the contribution of the subscale on the convective term:\\ $\quad\quad\quad\mathbf{R}^{n+1,i}-(\tilde{\mathbf{u}}^{n+1,i,k-1}\cdot\nabla\mathbf{u}_h^{n+1,i})\rightarrow\hat{\mathbf{R}}^{n+1,i,k-1}$\\
$\quad$ Compute $ \tau_1 $, $ \tau_{1,t} $ and $ \tau_2 $\\
$\quad$ Update the subscale:\\
$\quad\quad\quad\tilde{\mathbf{u}}^{n+1,i,k}=\tau_t\left(\frac{1}{\Delta t}\tilde{\mathbf{u}}^{n}+\hat{\mathbf{R}}^{n+1,i,k-1}\right)$\\
$\quad$ Check convergence\\
END
\end{tabular}}
\end{center}

With this algorithm for the update of the subscales, we have an explicit implementation of the residual projection. While for the global nonlinear iterations, the residual projection is treated implicitly.

Here is stated the algorithm implemented on FEMPAR.

\begin{center}
\fbox{
\begin{tabular}{l}
Read (or compute) $ \mathbf{u}_h^0 $ and set $ p_h^0=0 $, $ \tilde{\mathbf{u}}^0=\mathbf{0} $.\\
FOR $ n=0,...,N-1 $ DO:\\
$\quad$ Set $ i=0 $\\
$\quad$ Set $ \mathbf{u}_h^{n+1,0}=\mathbf{u}_h^n $, $ p_h^{n+1,0}=p_h^n $, $ \tilde{\mathbf{u}}^{n+1,0}=\tilde{\mathbf{u}}^n $\\
$\quad$ WHILE (not converged) DO:\\
$\quad\quad$ $ i\leftarrow i+1 $\\
$\quad\quad$ Set $ \mathbf{a}^{n+1,i-1}=\mathbf{u}_h^{n+1,i-1}+\tilde{\mathbf{u}}^{n+1,i-1} $\\
$\quad\quad$ For OSS, compute the projections $ \Pi(\mathbf{R}_{oss}^{n+1,i-1}) $ and $ \Pi(\nabla\cdot\mathbf{u}^{n+1,i-1}) $\\
$\quad\quad$ Compute $ \tau_1 $, $ \tau_{1,t} $ and $ \tau_2 $\\
$\quad\quad$ Compute $ \mathbf{u}_h^{n+1,i} $ and $ p_h^{n+1,i} $ by solving (\ref{3})\\
$\quad\quad$ Update the subscales $ \tilde{\mathbf{u}}^{n+1,i} $ using the previous nonlinear algorithm\\
$\quad\quad$ Check convergence\\
$\quad$ END\\
$\quad$ Set up the converged values $ \mathbf{u}_h^{n+1}=\mathbf{u}_h^{n+1,i} $, $ p_h^{n+1}=p_h^{n+1,i} $ and $ \tilde{\mathbf{u}}^{n+1}=\tilde{\mathbf{u}}^{n+1,i} $\\
END
\end{tabular}}
\end{center}